\section{\scAssesmentHeader{Норвегии}}


% \subsection{\scAssesmentBuildingClass}
% % 
% \subsection{\scAssesmentBuildingMetrics}

% ##########-----Регулирование


\subsection{\scAssesmentBuildingLaw}
% 
Регулирование вопросов энергоэффективности строительства в Норвегии закладывает основы на уровне законодательства о природопользовании.
С целю унификации процесса издаются методические материалы \cite{law_NORW_RulesCode_Natureuse} для владельцев земель и широкого круга участников градостроительной деятельности, включая частных застройщиков.


Kommuneplanes\footnote{Общинная (Коммуны являются вторым административным уровнем деления Норвегии после губерний  — фюльке)
            социальная часть плана определяет общие долгосрочные цели для сообщества Тромсё и муниципалитета Тромсё на двенадцатилетнюю перспективу и является стратегиейдля достижения этих целей.
            На этом уровне применяется механизм соучаствующего проектирования, частью которого являются публичные информационные системы (географический портал, сайт администрации общины),
            использующиеся в публичных обсуждениях и на ранних этапах принятия решений.
            На раннем этапе производится консультация с интересантами территории (преимущественно, это жители и владельцы предприятий)}
являются документами территориального планирования муниципального уровня, состоящих из социальной и землеустроительной частей муниципального генерального плана.
Ресурс www.gatami.no является порталом-агрегатором принятия решений, с помощью которого любой интересант, включая жителей,
может сообщать о проблемах и формировать запросы на развитие территории, а управленческая команда — получать оперативную информацию и проводить оперативную аналитику
и реагирование на ситуации. Прикладные информационные технологии позволяют создавать формы для пользователей и унифицировать структуры поступающих данных без создания избыточных юридически формализованных форм стандартизации. В свою очередь, унифицированная структура позволяет произвести автоматизированную обработку методами статистики, создавать визуализацию данных для управленческих команд и, как итог, повышать качество обратной связи и оперативность в принятии решений.
Представляющие ценность исторические здания являются предметом охраны местного законодательства об объектах культурного наследия.
Портал https://tromso.gravearbeider.no/ на базе геоинформационной платформы позволяет отслеживать заявки на археологические исследования территории застройки.

% ##########-----Опыт


\subsection{\scAssesmentExp}
