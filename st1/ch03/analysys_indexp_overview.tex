\section{\textbf{Изученность темы}}


Вопрос исследования традиционных подходов в строительстве коренного населения, а также влияние местной архитекторы,
культуры и быта на современные постройки является предметом изучения специалистов многих сфер, включая архитекторов- и градостроителей-практиков, а также исследователей в этой области.
Именно поэтому прежде чем приступать к собственному исследованию необходимо кратко осветить уже разработанные материалы и использовать полученный опыт предшественников.

Наиболее значимый вклад внесли следующие исследователи: К.Г. Туралысов, К.А. Лыткин, Б.А. Скупов, В. Шумовской …

В статье Бориса Скупова «Идеальный биоклиматический дом для Севера архитектора Клима Туралысова» описывается вклад якутского архитектора,
доктора архитектуры Клима Георгиевича Туралысова в изучение традиций строительства и разработку новаторских идей в сфере проектирования в экстремальных климатических условиях.

Далее кратко описаны основные моменты, которые подчеркиваются в статье [].

К.Г. Туралысов внес важный вклад в изучение особенностей устройства жилища коренного населения Якутии. Материалы его кандидатской диссертации, защищенной в 1980 году,
посвящены основам организации жилища для коренного сельского населения Якутии и в целом строительства жилых домов на севере \cite{1980bu_Turalysov_YakindugenosHabitat}.
Им была выявлена семейно-хозяйственная ячейка жилого здания, а также основные принципы ее организации.  

Докторская диссертация (1997 год) посвящена более широкой теме – концепции градостроительного освоения крупного северного региона в экстремальных климатических условиях,
которая рассматривается на примере арктической зоны Якутии – среды обитания народностей Севера \cite{1980bu_Turalysov_YakindugenosHabitat}.
В этой работе, а также в монографиях описывается концепция биоклиматического многофункционального северного жилища.

В своих трудах Клим Георгиевич проводит анализ опыта строительства местного населения в суровых северных условиях
и подчеркивает важность преемственности в приемах строительства и объемно-пространственной организации жилых построек.
Таким образом мы приходим к выводу, что для эффективного проектирования, строительства и самое главное эксплуатации построек необходимо опирать на опыт коренного населения,
которое успешно выживает в сложнейших климатических условиях уже не одну сотню лет.

Опираясь на исторические реконструкции, К.Г. Туралысов говорит о том, что «предшественником» национального жилища якутов была юрта кочевников.
Однако в связи с особенностями климата, а также ведения хозяйства было нецелесообразно воспроизводить ее на просторах тундры,
поэтому образ юрты трансформировался в урасу – одну из древнейших форм якутского жилища.
Далее прослеживается преемственность формы в более поздних шестиугольных деревянных строениях.

Вышеперечисленные 

Таким образом мы приходим к выводу, что жилища коренного населения, так и переселенцев отвечало требованиям минимальной теплопотери,
а потому было приближено в плане к кругу для максимального сохранения тепла.

Практически параллельно исследованиям К.Г. Туралысова В. Шумовской, изучая особенности и поведение различных материалов в условиях Севера,
развивал идею шестиугольного здания моносотной структуры – дома-моносоты. Правильная форма здания в плане, тупые углы помогают накапливать и сохранять тепло внутри.
Поставленный на сваи, дом становится быстровозводимым, а также отвечает требованиям надежности, т.к. позволяет нивелировать воздействие многолетней мерзлоты и пучения грунтов.

Компактность описанного здания также отвечает требованиям по сопротивлению воздействия сурового климата.
Американским архитектором Ральфом Ноулзом была доказана концепция того, что зависимость подверженности здания воздействию климата,
прямо пропорциональна отношению его площади к объему. То есть чем меньше площадь поверхности здания при том же объеме, тем меньше его уязвимость при воздействии климата.


