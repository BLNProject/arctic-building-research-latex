\section{Общемировой опыт и парадигма реализации энергоэффективных зданий}


Сокращение добычи природных энергетических ресурсов и, соответственно увеличение их стоимости, с начала XXI века является вызовом для отрасли жилищного строительства в частности и
для проектирования мер развития территорий в целом. Поскольку этот вызов формирует запрос на оптимизацию энергопотребления при развитии территорий во всем мире,
не ограничиваясь арктическими широтами, авторы в данном вводном разделе рассматривают общемировой опыт реализации энергоэффективных зданий в разрезе формирования
общей парадигмы и принципиального подхода к стандартизации процессов энергоэффективного жилого строительства.

Впервые государственную политику в отношении энерго-ресурсосбережения начали проводить в Соединенных Штатах Америки в начале 90-х годов прошлого столетия.
Затем, ряд документов, так называемых <<конвенций>> --- соглашений, были приняты в странах ЕС.
Это были, в первую очередь, государственные законы, стимулирующие внедрение энергосберегающих технологий \cite{2010buee_Sormunen_Finland,2020buee_Sheyna_HabitatEurotechInsertion}. % Возможно найти первоисточник — сами законы?

В странах Европы, Канаде и Соединенных Штатах Америки энергоэффективное мышление при проектировании и эксплуатации зданий имеет наибольшее развитие:
уделяется внимание энергосберегающим процессам и инструментам их реализации на законодательном уровне, на уровне методологии и стандартизации.
В этих странах строят не просто энергоэффективные здания, а так называемые <<зеленые здания>>, к которым применяются регламентируемые по нормам повышенные требования безопасности,
снижения негативного воздействия на окружающую среду.
<<Зеленые здания>>, согласно требованиям норм, способствуют комфортному проживанию людей и учитывают интересы будущего поколения \cite{2020buee_Muhammad_USWastemanagement}.
Проектирование экологически устойчивых зданий, <<зеленое>> эко-проектирование (Green Architecture) \cite{2008buee_wines_green} и технологические инновации в данной области --- одно из самых популярных и
развиваемых направлений.% Канада --- один из лидеров среди них.% Найти подтверждения — статистику и пр. — что Канада — лидер
Так как практически половина потребления всей получаемой от глобальных энергоресурсов энергии % Уточнить цитату и концепты по указанному источнику
приходится на жилые дома и сооружения, то одним из самых очевидных методов ресурсосбережения становится строительство энергоэффективных и пассивных зданий \cite{1979buee_mazria_passive}.


% Обзор систем стандартизации энергоэффективности — описание текстом, матрица
% ##########################################################################################

% TO DO: таблица стандартов с шапкой:
% Наименование стандарта & Юрисдикция (страны действия) & Способ применения (добровольно/как часть НПА) & Описание метрик (что оценивается) & Описание модели (как оценивается)
\begin{landscape}
    
    % \KOMAoptions{paper=landscape}
    % \recalctypearea
        \begin{center}
            \begin{longtable}{|m{40mm}|p{40mm}|p{40mm}|p{55mm}|p{55mm}|}
               \caption{Общемировые стандарты и системы рейтинговой оценки энергетической эффективности зданий}
                \label{tab:review_energystandarts}
                \\ \hline
                % \multirow{2}{20mm}{№} &
                %     \multirow{2}{38mm}{Наименование улуса (района)} &
                %     \multirow{2}{38mm}{Улусный (районный) центр} &
                %     ГСОП, °С $\times$сут &
                %     \multicolumn{4}{c|}{$R_\text{огрконструкции}^\text{требуемое}\text{, м}{^2} \times \text{°С/Вт}$} \\
                % \cline{5-8}
                % &   &  & &  стен    &
                %     покрытий и перекрытий над проездами &
                %     Перекрытий чердачных над неотапливаемыми подпольями и подвалами &
                %     окон и балконных дверей, витрин и витражей  \\
                Наименование \mbox{стандарта} & Юрисдикция (страны действия) & Способ применения (добровольно/как часть НПА) &
                    Описание метрик (что оценивается) & Описание модели (как оценивается) \\

    
                \hline \endfirsthead
                % \cline{1-10}
                \subcaption{Продолжение таблицы~\ref{tab:review_energystandarts}}
                \\ \hline \endhead
                \hline \subcaption{Продолжение на след. стр.}
                \endfoot
                \hline \endlastfoot

                        % &       &   &    &
                BREEAM (\mbox{англ.}~Building  Research Establishment Environmental Assessment Method): рейтинговая  система  оценки «зеленых» зданий,разработанная в 1990~г. британской организацией BRE Global. &
                    &   &   & \\ \hline
                LEED (\mbox{англ.}~The Leadership in Energy \& Environmental Design): рейтинговая система оценки «зеленых» зданий, разработанная в 1998~г. Американским советом USGBC. &
                    &   &   & \\ \hline
                DGNB (\mbox{нем.}~Deutsche Gesellschaft fur Nachhaltiges Bauen): рейтинговая система оценки «зеленых» зданий, разработанная в 2007~г. Немецким советом по устойчивому строительству. &
                    Действует на территории стран Евросоюза &
                    Система DGNB оценивает не отдельные показатели, а общую производительность здания на основе критериев.
                    Если эти критерии выполняются превосходно, здание получает сертификат или предварительный сертификат (платина, золото, серебро или бронза) для существующей недвижимости.
                    DGNB продолжает развивать свою систему сертификации и адаптирует ее к национальным и международным стандартам и законодательству,
                    и, преемствуя парадигму системы Eurocode, носит рекомендательный характер &
                    Оценка производится по критериям, которые синхронизированны с 17-ю целями устойчивого развития ООН \cite{UN_17Goals} и техническими регламентами среды обитания человека,
                    закреплёнными в межгосударственных стандартах Eurocode. Критерии выделяются в следующие группы --- в соответствии с отраслями жизнедеятельности человека и предметными областями,
                    задействованными в развитии территорий:
                    \begin{enumerate}[1)]
                        \item Environmental quality (Качество среды) --- шесть критериев качества окружающей среды позволяют проводить оценку воздействия зданий на глобальную и местную окружающую среду, а также воздействия на ресурсы и образования отходов;
                        \item Economic quality (Качество хозяйственной деятельности) --- три критерия оценки долгосрочной экономической жизнеспособности (издержек жизненного цикла) и экономического развития;
                        \item Sociocultural and functional quality (Качество социокультурного и программного наполнения (среды)) --- восемь критериев социокультурного и функционального качества помогают оценивать здания с точки зрения здоровья, комфорта и удовлетворенности пользователей, а также основных аспектов функциональности;
                        \item Technical quality (Качество технических решений) --- семь критериев обеспечивают шкалу оценки технического качества с учетом соответствующих аспектов устойчивости;
                        \item Process quality (Качество процессов) --- девять критериев качества процесса направлены на повышение качества планирования и обеспечение качества строительства;
                        \item Site quality (Качество (строительной) площадки) --- четыре критерия качества площадки оценивают взаимовлияние окружающуей среды и проекта.
                    \end{enumerate} &
                    Оценка производится на основе показателей, описанных в межгосударственных стандартах Eurocode, действующих на территории Евросоюза.
                    Стандарты EN Eurocode делятся по аналогичным, принятым в системе DGNB, группам, и носят рекомендательный характер.
                    В некоторые стандарты закладываются математические модели, позволяющие провести физические расчёты и сформировать интегральные показатели строительно-инвестиционного проекта
                    в рамках Feasibility Studies \footnote{Технико-экономическое обоснование, проводимое на основе информационных моделей, полученных в ходе инженерных изысканий участка и градостроительного анализа территории на предмет ограничений и правового режима земельных участков.
                            В строительной практике США, стран Евросоюза и ряда других развитых стран является основным этапом принятия проектных решений и основой разработки строительной (<<рабочей>>) документации.
                            На этом этапе интегральные показатели информационной модели целостно охватывают проектируемый объект в контексте окружающей среды.}
                    с применением технологий информационного моделирования площадки строительства для принятия решений
                    Обязательные положения устанавливаются каждым из государств индивидуально, в соответствии с принципами гражданского общества и местного самоуправления \\ \hline

                % \hline
            \end{longtable}
        \end{center}
    
    % \KOMAoptions{paper=seascape}
    % \recalctypearea
\end{landscape}
% С января 2022-го ратифицированная версия действует и в РФ
Существенным отличием в порядке проектирования энергоэффективных зданий от нашей страны является тот факт, что на территории Канады и США действует
специальная рейтинговая система сертификации: The Leadership in Energy \& Environmental Design (LEED)\footnote{рус. <<Лидерство в Энергетическом и Средовом Проектировании>>} \cite{method_US_LEED} .
Система LEED была разработана в 1998 году United States Green Building Council (USGBC) как стандарт измерения проектов энергоэффективных,
экологически чистых и устойчивых зданий для осуществления перехода строительной индустрии к проектированию, строительству и эксплуатации таких зданий.
Важно отметить, что LEED не заменяет собой требования нормативных документов, установленных в той или иной стране государственными ведомствами,
она только дополняет более совершенными, отвечающими запросам современности, критериями оценки качества.

Новый подход помогает решить такие задачи, как снижение уровня потребления энергетических и материальных ресурсов зданием,
снижение неблагоприятного воздействия на природные эко-системы, обеспечение гарантированного уровня комфорта среды обитания человека,
создание новых энергоэффективных и энергосберегающих продуктов, новых рабочих мест в производственном и эксплуатационном секторах,
формирование общественной потребности в новых знаниях и технологиях в области возобновляемой энергетики, формирование у проектировщиков ответственность за
эффективность решений и будущие функции систем \cite{method_US_LEED}.

Кроме законодательных требований к энергоэффективности зданий, государствами используется метод поощрения и стимулирования энергосбережения:
\begin{itemize}
    \item Германия ежегодно выделяет субсидии для реконструкции зданий с использованием энергоэффективных технологий;
    \item В Канаде строительство таких зданий всячески поддерживается на уровне правительства: для эко- проектирования предусмотрены льготные условия и финансовые стимулы в законодательстве;
    \item В Швейцарии инвесторы могут вкладывать денежные средства в строительство зданий с низким энергопотреблением и получить грант от государства;
    \item Во Франции жильцы домов, сданных до 1977 г., решившие утеплить здание --- могут получить скидку и снижение налоговой ставки до 40\% \cite{2016buee_Sheyna_AltenergyWorldtrands}.
\end{itemize}

% Краткие выводы по моделям ИЛИ перенос в выводы по Главе
% ##########################################################################################

% Перенести в опыт применения одной из стран ИЛИ в \section{Общеприменимые в мире строительные технологии}, если таковой потребуется создать
В европейских странах разрабатываются разного рода здания с повышенной энергоэффективностью и пониженным энергопотреблением направленных к уровню <<Zero Energy Building>> \cite{2017buee_MORCK_240}.
В зарубежном опыте при исследовании зданий на дефекты, ухудшающие тепловую эффективность здания, применяются разного рода тепловизионные устройства в режиме <<time-lapse>>,
при котором заметны все тонкости изменения температурного режима здания после изменения температуры наружной окружающей среды \cite{2015bueemethod_FOX_95,2016bueemethod_FOX_317}.
В работе \cite{2014buee_MEMON_870} описываются способы и методы интеграции энергоемких материалов в строительных конструкциях.


Сам факт использования таких материалов имеет хороший потенциал для зданий при сокращении энергопотребления, но требует некоторых инвестиций во время строительства.
Инвестирование и затраты на повышение тепловой эффективности здания выгодны по экологическим и почти по всем экономическим критериям \cite{2017buee_ADAMCZYK_421}.
В данном случае описываются затраты на потребление 1 кВт*ч в здании, при котором у здания со слабой тепловой защитой повышаются затраты на потребление энергетических ресурсов. 
В Китае нормативом, использующимся для оценки гражданских <<зеленых>> зданий, является <<GB-T 50378-2019 Стандарт оценки зеленых зданий>>,
который является переизданным изданием аналогичного норматива 2014 года с повышением требований.


% Перенести в Таблицу стандартов
Система индексов оценки <<зеленых>> зданий включает пять типов показателей: безопасность и долговечность, здоровье и комфорт, удобство проживания,
уровень ресурсосбережения и комфортность окружающей среды. Каждый показатель имеет обязательные и дополнительные требования для получения баллов \cite{2020buee_Gushin_EnergysaveTrends}.




% Перенос в методы регулирования РФ


