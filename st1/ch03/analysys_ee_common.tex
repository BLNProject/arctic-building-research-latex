\section{Современные требования по тепловой защите зданий}


Сокращение добычи природных энергетических ресурсов и соответственно увеличение их стоимости является давно известной проблемой во всем мире.
Впервые государственную политику в отношении энерго-ресурсосбережения начали проводить в Соединенных штатах Америки в начале 90-х годов прошлого столетия,
затем ряд документов, так называемых конвенций, были приняты в странах ЕС.
Это были в первую очередь государственные законы, стимулирующие внедрение энергосберегающих технологий \Code{[1, 2]}. 

В странах Европы, Канаде и Соединенных Штатах Америки уделяется особенное внимание на энергоэффективность зданий и на методы повышения ее уровня.
В этих странах строят не просто энергоэффективные здания, а так называемые <<зеленые здания>>, к которым регламентируются по западным нормам повышенные требования по безопасности,
негативному воздействию на окружающую среду, также здания должны способствовать комфортному проживанию людей и учитывать интересы будущего поколения \Code{[3]}.
Проектирование экологически устойчивых зданий, <<зеленое>> эко-проектирование (Green Architecture) \Code{[4]} и технологические инновации в данной области - одно из самых популярных и
развиваемых направлений. Канада – занимает одно из первых мест среди них. Так как практически половина потребления всей получаемой от глобальных энергоресурсов энергии
приходится на жилые дома и сооружения, то одним из самых очевидных методов ресурсосбережения становится строительство энергоэффективных и пассивных	зданий \Code{[5]}. 
Существенным отличием в порядке проектирования энергоэффективных зданий от нашей страны является тот факт, что на территории Канады и США действует
специальная рейтинговая система сертификации: The Leadership in Energy \& Environmental Design (LEED) \Code{[6]} – в переводе <<Лидерство в энергетическом и экологическом проектировании>>.
Система LEED была разработана в 1998 году United States Green Building Council (USGBC) как стандарт измерения проектов энергоэффективных,
экологически чистых и устойчивых зданий для осуществления перехода строительной индустрии к проектированию, строительству и эксплуатации таких зданий.
Важно отметить, что LEED не заменяет собой требования нормативных документов, установленных в той или иной стране государственными ведомствами,
она только дополняет более совершенными, отвечающими запросам современности, критериями оценки качества.

Новый подход помогает решить такие задачи, как снижение уровня потребления энергетических и материальных ресурсов зданием,
снижение неблагоприятного воздействия на природные эко-системы, обеспечение гарантированного уровня комфорта среды обитания человека,
создание новых энергоэффективных и энергосберегающих продуктов, новых рабочих мест в производственном и эксплуатационном секторах,
формирование общественной потребности в новых знаниях и технологиях в области возобновляемой энергетики, формирование у проектировщиков ответственность за
эффективность решений и будущие функции систем \Code{[6]}.
Кроме законодательных требований к энергоэффективности зданий зарубежными странами используется метод поощрения и стимулирования энергосбережения.
Так Германия ежегодно выделяет субсидии для реконструкции зданий с использованием энергоэффективных технологий.
В Канаде строительство таких зданий всячески поддерживается на уровне правительства: для эко- проектирования предусмотрены льготные условия и финансовые стимулы в законодательстве.
В Швейцарии инвесторы могут вкладывать денежные средства в строительство зданий с низким энергопотреблением и получить грант от государства.
Во Франции жильцы домов, сданных до 1977 г. и решившие утеплить зданий, могут получить скидку и снижение налоговой ставки до 40\% \Code{[7]}.
Таким образом, опираясь на опыт и принципы энергосбережения зарубежных стран, кроме создания строгих законодательных документов повышения энергоэффективности зданий необходимо внедрить систему поощрения и стимулирования использования энергоэффективных технологий как юридических, так и физических лиц.
В европейских странах разрабатываются разного рода здания с повышенной энергоэффективностью и пониженным энергопотреблением направленных к уровню <<Zero Energy Building>> \Code{[8]}.
В зарубежном опыте при исследовании зданий на дефекты, ухудшающие тепловую эффективность здания, применяются разного рода тепловизионные устройства в режиме <<time-lapse>>,
при котором заметны все тонкости изменения температурного режима здания после изменения температуры наружной окружающей среды \Code{[9, 10]}.
В работе \Code{[11]} описываются способы и методы интеграции энергоемких материалов в строительных конструкциях.
Сам факт использования таких материалов имеет хороший потенциал для зданий при сокращении энергопотребления, но требует некоторых инвестиций во время строительства. Инвестирование и затраты на повышение тепловой эффективности здания выгодны по экологическим и почти по всем экономическим критериям \Code{[12]}. В данном случае описываются затраты на потребление 1 кВт*ч в здании, при котором у здания со слабой тепловой защитой повышаются затраты на потребление энергетических ресурсов. 
В Китае нормативом, использующимся для оценки гражданских <<зеленых>> зданий, является <<GB-T 50378-2019 Стандарт оценки зеленых зданий>>,
который является переизданным изданием аналогичного норматива 2014 года с повышением требований. Система индексов оценки <<зеленых>> зданий включает пять типов показателей: безопасность и долговечность, здоровье и комфорт, удобство проживания, уровень ресурсосбережения и комфортность окружающей среды. Каждый показатель имеет обязательные и дополнительные требования для получения баллов \Code{[13]}.
В Российской Федерации с начала 90-х годов прошлого столетия начали серьезно заниматься созданием новой нормативно-правовой базы для повышения энергоэффективности зданий и
в 1995 году впервые было введено нормирование приведенного сопротивления теплопередаче ограждающих конструкций в зависимости от градусо-сутки отопительного периода
и введены изменения в СНиП II-3-79* <<Строительная теплотехника>> \Code{[14]}.

Повышение требований по тепловой защите зданий разделили на два этапа: 1998-2000 гг. и 2001-2005 гг.
Для примера в Таблице \ref{tab:method_gsop_phases} приведены требуемые значения приведенного сопротивления теплопередаче стен, цокольного перекрытия и окон для жилых зданий в условиях г. Якутска.
Также 3 апреля 1996 г. был принят Федеральный закон от N 28-ФЗ "Об энергосбережении" и началась реализация федеральной целевой программы "Энергосбережение России" на 1998 - 2005 годы.
% \begin{equation}
%     R_\text{огрконструкции}^\text{требуемое}\text{, м}{^2} \times \text{°С/Вт}
% \end{equation}

\begin{center}
    \begin{longtable}{|m{0.26\textwidth}|c|c|c|}
        \caption{Требуемые значения приведенного сопротивление теплопередаче конструкций для многоквартирных зданий в г.Якутске по СНиП II-3-79* <<Строительная теплотехника>> с изменениями № 3}
        \label{tab:method_gsop_phases}
        \\ \hline
        \multirow{2}{8cm}{ГСОП, \textdegree C$\times$сут\slash год} & \multicolumn{3}{c|}{$R_\text{огрконструкции}^\text{требуемое}\text{, м}{^2} \times \text{°С/Вт}$}\\
        \cline{2-4}
        & Стены & Покрытия, цокольное перекрытие & Окна \\
        \hline \endfirsthead
        \subcaption{Продолжение таблицы~\ref{tab:method_gsop_phases}}
        \\ \hline \endhead
        \hline \subcaption{Продолжение на след. стр.}
        \endfoot
        \hline \endlastfoot
        1 этап & 2,91 & 4,80 & 0,76             \\
        \hline
        2 этап & 5,09 & 7,48 & 0,76             \\
        \hline
    \end{longtable}
\end{center}

В 2003 г. введен в действие СНиП 23-02-2003 <<Тепловая защита зданий>> \cite{law_RU_Rules_Code_ThermalPerformance} и в нем предусмотрен дополнительно комплексный подход, заключавшийся в ограничении затрат тепла на отопление и вентиляцию. При выполнении указанных условий разрешалось некоторое ослабление требований к отдельным ограждающим элементам. 
В ноябре 2009 года принят Федеральный закон №261 <<Об энергосбережении и о повышении энергетической эффективности…>> \cite{law_RU_fz_EnergyEff}.
В соответствии с этим законом, предусмотрено снижение энергоемкости ВВП России на 40\% к 2020 году и в 2.5. – 3 раза к 2030 году (относительно уровня 2007 года).
В дальнейшем НИИСФ РААСН проведены научно-прикладные работы по гармонизации российских норм по тепловой защите зданий с аналогичными зарубежными нормами развитых стран \Code{[16]}.
В настоящее время оценка новой нормативно-правовой базы показала, что Россия серьезно улучшила позицию в рейтинге среди стран по реализации политики энергоэффективности.
При проектировании тепловой защиты зданий и сооружений \Code{[17]}.

Главным регламентирующим документом выступает СП 50.13330.2012 <<Тепловая защита зданий. Актуализированная редакция СНиП 23-02-2003>> \cite{law_RU_Rules_Code_ThermalPerformance}.
Основные характеристики климата района строительства объектов определяются по СП 131.13330.2020 <<Актуализированная редакция СНиП 23-01-99* Строительная климатология>> \cite{law_RU_RulesCode_BuildingClimatology}.
Основные параметры микроклимата в помещениях зданий регламентируются ГОСТ 30494-2011 <<Здания жилые и общественные. Параметры микроклимата в помещениях>> \cite{law_RU_RulesCode_BuildingMicroclimateResedentialPublic}.

В строительных нормах СП 50.13330.2012 \cite{law_RU_Rules_Code_ThermalPerformance} установлены требования к:
\begin{itemize}
    \item приведенному сопротивлению теплопередаче ограждающих конструкций здания;
    \item удельной теплозащитной характеристике здания;
    \item ограничению минимальной температуры и недопущению конденсации влаги на внутренней поверхности ограждающих конструкций в холодный период года, за исключением светопрозрачных конструкций с вертикальным остеклением (с углом наклона заполнений к горизонту 45° и более);
    \item теплоустойчивости ограждающих конструкций в теплый период года;
    \item воздухопроницаемости ограждающих конструкций;
    \item влажностному состоянию ограждающих конструкций;
    \item теплоусвоению поверхности полов;
    \item расходу тепловой энергии на отопление и вентиляцию зданий.
\end{itemize}


% Рассмотрим некоторые требования из них относительно жилых зданий, строящихся в арктических районах Республики Саха (Якутия). 
% Для проектирования тепловой защиты в соответствие с СП 50.13330.2012 \cite{law_RU_Rules_Code_ThermalPerformance} определяется основной показатель - градусо-сутки отопительного периода (1) в зависимости от места расположения объекта строительства:

\begin{eqndesc}
    \begin{equation}\label{eq:gsop}
        \text{ГСОП}=\ (t_{\text{внутр}}-t_{\text{отпериода}})\times z_{\text{отпериода}}
    \end{equation}

    где $t_{\text{внутр}}$ — расчётная температура внутреннего воздуха, °С,\\
    $t_{\text{отпериода}}$ — средняя температура периода со средней суточной температурой воздуха ниже или равной 8°С (в соответствии с положениями \cite{law_RU_RulesCode_BuildingClimatology}),\\
    $z_{\text{отпериода}}$ — продолжительность (в сутках) периода со средней суточной температурой воздуха ниже или равной 8°С (в соответствии с положениями \cite{law_RU_RulesCode_BuildingClimatology}).
\end{eqndesc}

% ГСОП=(t_в-t_от)×z_от,           (1)
% где t_в – расчетная температура внутреннего воздуха здания, °С;
% t_от, z_от - средняя температура наружного воздуха, °С, и продолжительность, сут/год, отопительного периода.


В Таблице \ref{tab:pethod_gsop_calcvalues} приведены основные климатические параметры холодного периода года для районных центров арктических улусов республики.
Климатические параметры установлены в соответствии с СП 131.13330.2020 <<Актуализированная редакция СНиП 23-01-99* Строительная климатология>> \Code{[19]}.
Для 7 населенных пунктов арктических районов: пп.Белая гора, Чокурдах, Тикси, Черский, Депутатский,
сс.Хонуу и Батагай-Алыта климатические параметры в СП 131.13330.2020 \Code{[19]} отсутствуют и в Таблице \ref{tab:pethod_gsop_calcvalues} приняты из данных,
использованных при разработке ТСН 23-343-2002 РС(Я) <<Теплозащита и энергопотребление жилых и общественных зданий>>.

В дальнейшем требуется уточнение климатических параметров для основных населенных пунктов арктических районов с учетом метеорологических данных и обширной территории.
Из представленных в таблице 1.2 параметров видно, что арктические районы характеризуются самыми экстремальными климатическими условиями:
\begin{itemize}
    \item количество дней в году с температурой наружного воздуха ниже -40°С составляет– 60-80 дней и установлена абсолютно минимальная температура от 68°С до -58°С;  
    \item продолжительность суток со средней температурой воздуха -8°С и ниже составляет от 266 до 365 дней;
    \item средняя температура за отопительный период от -13,4°С до -24,7°С;
    \item градусо-сутки отопительного периода ГСОП варьируется от 10906 до 12556°С·сут/год;
    \item максимальная скорость ветра в январе превышает в большинстве районов превышает 2,5 м/сек.
\end{itemize}

В соответствии \Code{[5]} из расчета градусо-сутки отопительного периода на территории арктических районов Республики Саха (Якутия)
установлены повышенные требования по тепловой защите зданий. Нормируемое значение приведенного сопротивления теплопередаче ограждающей конструкции определяются
в соответствии с СП 50.13330.2012 \Code{[5]} определяется по формуле:

\begin{eqndesc}
    \begin{equation}\label{eq:heatmoveresist}
        R_\text{огрконструкции}^\text{нормируемое}=\ R_\text{огрконструкции}^\text{требуемое}\times m_\text{регионкоэфф}
    \end{equation}

    где $R_\text{огрконструкции}^\text{требуемое}$ -- базовое значение требуемого сопротивления теплопередаче ограждающей конструкции, м$^2 \times$ °С\slash Вт, следует принимать в зависимости от градусо-суток отопительного периода, ГСОП, °С$\times$сут\slash год , региона строительства и определять по таблице 3 \Code{[5]}, \\
    $m_\text{регионкоэфф}$ -- коэффициент, учитывающий особенности региона строительства. В расчете по формуле \eqref{eq:heatmoveresist} принимается равным 1.
\end{eqndesc}

% R_о^норм=R_о^тр∙m_(p ),          (2)
% где R_о^тр - базовое значение требуемого сопротивления теплопередаче ограждающей конструкции, м2∙°С/Вт, следует принимать в зависимости от градусо-суток отопительного периода, ГСОП, °С∙сут/год , региона строительства и определять по таблице 3 \Code{[5]};
% m_(p )- коэффициент, учитывающий особенности региона строительства. В расчете по формуле (2) принимается равным 1.

Допускается снижение значения коэффициента $m_\text{регионкоэфф}$ в случае если при выполнении расчета удельной характеристики расхода тепловой энергии на отопление и
вентиляцию здания по методике Приложения Г \Code{[5]} выполняются требования п. 10.1 \Code{[5]} к данной удельной характеристике. 
Результаты многочисленных исследований показывают, что потери тепловой энергии через наружные ограждающие конструкции являются основными и составляют более 50\%
в структуре затрат тепловой энергии здания на отопление в течение отопительного периода \Code{[22, 23]}.
В СП 50.13330.2012 \cite{law_RU_Rules_Code_ThermalPerformance} реальное ужесточение поэлементных требований заключается в нормировании приведенного сопротивления теплопередаче ограждающих конструкций,
отражающего влияние плоских, линейных и точечных теплопроводных включений. В качестве обязательного приложения представлен значительно модернизированный метод расчета приведенного сопротивления теплопередаче ограждающих конструкций.
При этом расчетная теплопроводность всех строительных материалов, применяемых в ограждающих конструкциях, принимается с учетом их эксплуатационной влажности \Code{[24]}.
Теплопроводные включения значительно снижают теплозащиту зданий и требуют применения различных конструктивных мероприятий при проектировании зданий.
В дополнение к действующему нормативному документу НИИСФ в \Code{[25]} определены и приведены теплотехнические характеристики различных ограждающих конструкций и
их узлов в зависимости от физических и геометрических параметров имеющихся неоднородностей.
Следует отметить, что в европейских странах теплопроводные включения нормируются отдельно и в большинстве случаев учет их
влияния при проектировании ограждающих конструкцийпроизводится не в полной мере \Code{[26]}.
Для точного учета теплопроводных включений предлагаются различные методики расчета приведенного сопротивления теплопередаче ограждающих конструкций \Code{[27, 28]}.
В таблице 1.3 приведены базовые значения требуемого сопротивления теплопередаче Rтрo для ограждающих конструкций жилых зданий в арктических районах.
Следует отметить, что при значениях ГСОП более 10000°С$\times$сут\slash год для климатических условий арктических районов требования по сопротивлению теплопередаче для
наружных ограждающих конструкций весьма высоки. Так, требуемые сопротивления для стен составляют 5,22 м$^2 \times$°С\slash Вт и выше,
покрытий и цокольного перекрытия - 7,65 м$^2 \times$°С\slash Вт и выше, окон и балконных дверей – 0,77 м$^2 \times$°С\slash Вт и выше.

Одним из важных нормируемых показателей теплозащитных свойств ограждающих конструкций является температурный перепад между температурой внутреннего воздуха
и температурой на внутренней поверхности $\Delta t_\text{н}$, °С, который составляет для жилых зданий:

\begin{itemize}
    \item наружных стен 4,0°С;
    \item цокольных перекрытий 2,0°С.
\end{itemize}

В условиях Крайнего Севера особенно сложно выполнить данное требование для пола первого этажа жилого дома с проветриваемым холодным подпольем.
В условиях особо низкой температуры наружного воздуха и проветриваемого подполья значительное влияние на теплозащиту нижних этажей
многоэтажных зданий оказывает инфильтрация воздуха. В период наиболее холодных месяцев ($t_\text{н}$=-40°С…-55°С) с учетом ветра
на первом этаже разница давления воздуха на наружном и внутреннем поверхностях ограждающих конструкций в условиях арктических районов составляет:
\begin{itemize}
    \item для одноэтажного дома при высоте от уровня пола первого этажа до верха вытяжной шахты 5,0 м --- 12,85÷18,6 Па;
    \item для двухэтажного дома при высоте от уровня пола первого этажа до верха вытяжной шахты 8,0 м --- 18,97÷25,02 Па.
\end{itemize}

% Таблица 1.2 - Расчетные параметры температуры по арктическим улусам РС(Я)

% \clearpage

% \begin{landscape}
    
% % \KOMAoptions{paper=landscape}
% % \recalctypearea
%     \begin{center}
%         \begin{longtable}{|m{0.02\textwidth}|p{0.2\textwidth}|p{0.2\textwidth}|p{0.12\textwidth}|p{0.13\textwidth}|p{0.12\textwidth}|p{0.14\textwidth}|p{0.14\textwidth}|p{0.14\textwidth}|p{0.14\textwidth}|}
%            \caption{Расчетные параметры температуры по арктическим улусам РС(Я)}
%             \label{tab:pethod_gsop_calcvalues}
%             \\ \hline
%                 \multirow{2}{18cm}{№} &
%                 \multirow{2}{18cm}{Наименование\\ улуса\\ (района)} &
%                 \multirow{2}{18cm}{Улусный \\ (районный)\\ центр} &
%                 \multirow{2}{18cm}{Абсолютно\\ минималь- ная температура\\ воздуха, °С} &
%                 \multirow{2}{18cm}{Температура\\ воздуха\\ наиболее\\ холодной\\ пятидневки,\\ °С, с обес-\\ печен- ностью 0,92} &
%                 Продолжи- тельность, сут, и средняя температура воздуха, °С, периода со средней суточной температурой воздуха $\leqslant$ 8°С &       &
%                 \multirow{2}{18cm}{Количество\\ осадков\\ ноябрь-март,\\ мм} &
%                 \multirow{2}{18cm}{Преоблада-\\ ющее\\ направление ветра\\ декабрь-февраль} &
%                 \multirow{2}{18cm}{Максималь-\\ ная из скоростей\\ ветра по румбам\\ за январь, м/с} \\
            
                
%             \cline{6-7}
%              & & & & & Продолжи- тельность & Средняя температура  & &  \\

%             \hline \endfirsthead
%             % \cline{1-10}
%             \subcaption{Продолжение таблицы~\ref{tab:method_gsop_calcvalues}}
%             \\ \hline \endhead
%             \hline \subcaption{Продолжение на след. стр.}
%             \endfoot
%             \hline \endlastfoot
%             1 & Абыйский             & п. Белая гора            & -52 & 282 & 21,0 &    &       &       \\ \hline
%             2 & Аллаиховский         & п. Чокурдах              & -49 & 318 & 17,5 &    &       &       \\ \hline
%             3 & Анабарский           & с. Саскылах              & -60 & -53 & -19,0 & 51 & ЮВ   & 3,4   \\ \hline
%             4 & Булунский            & п. Тикси                 & -44 & 365 & 13,4 &    &       & 7,7   \\ \hline
%             5 & Верхнеколымский      & п. Зырянка               & -59 & -50 & -20,0 & 82 & С    & 3,0   \\ \hline
%             6 & Верхоянский          & п. Верхоянск             & -68 & -58 & -24,7 & 35 & ЮЗ   & 1,4   \\ \hline
%             7 & Жиганский            & с. Жиганск               & -60 & -52 & -19,8 & 69 & Ю    & 4,1   \\ \hline
%             8 & Момский              & с. Хонуу                 & -58 & 275 & 23,5 &    &       &       \\ \hline
%             9 & Нижнеколымский       & п. Черский               & -48 & 296 & 16,8 &    &       &       \\ \hline
%             10 & Оленекский          & с. Оленек                & -63 & -55 & -18,7 & 67 & В    & 2,6   \\ \hline
%             11 & Среднеколымский     & г. Средне-колымск        & -58 & -50 & -19,4 & 72 & ЮЗ   & 2,3   \\ \hline
%             12 & Усть-Янский         & п. Депутатский           & -53 & 293 & 21,4 &    &       &       \\ \hline
%             13 & Эвено-Бытантайский  & с. Батагай-Алыта         & -55 & 295 & 20,7 &    &       &       \\ 

%             \hline
%         \end{longtable}
%     \end{center}

% % \KOMAoptions{paper=seascape}
% % \recalctypearea
% \end{landscape}


\begin{landscape}
    
% \KOMAoptions{paper=landscape}
% \recalctypearea
    \begin{center}
        \begin{longtable}{|m{5mm}|p{35mm}|p{30mm}|p{22mm}|p{22mm}|p{22mm}|p{22mm}|p{22mm}|p{22mm}|p{22mm}|}
           \caption{Расчетные параметры температуры по арктическим улусам РС(Я)}
            \label{tab:method_gsop_calcvalues}
            \\ \hline
            № &
                Наименование улуса (района)&
                Улусный (районный) центр &
                Абсолютно минимальная температура воздуха, °С &
                Температура воздуха наиболее холодной пятидневки, °С, с обеспеченностью 0,92 &
                Продолжи-\newline тельность, сут, периода со средней суточной температурой воздуха $\leqslant$8°С &
                Cредняя температура воздуха, °С, периода со средней суточной температурой воздуха $\leqslant$8°С &
                Количество осадков ноябрь-март, мм &
                Преоблада-\newline ющее направление ветра декабрь-февраль &
                Максималь-\newline ная из скоростей ветра по румбам за январь, м/с \\

            \hline \endfirsthead
            % \cline{1-10}
            \subcaption{Продолжение таблицы~\ref{tab:method_gsop_calcvalues}}

            \\ \hline
            № &
                Наименование улуса (района)&
                Улусный (районный) центр &
                Абсолютно минимальная температура воздуха, °С &
                Температура воздуха наиболее холодной пятидневки, °С, с обеспеченностью 0,92 &
                Продолжи-\newline тельность, сут, периода со средней суточной температурой воздуха $\leqslant$8°С &
                Cредняя температура воздуха, °С, периода со средней суточной температурой воздуха $\leqslant$8°С &
                Количество осадков ноябрь-март, мм &
                Преоблада-\newline ющее направление ветра декабрь-февраль &
                Максималь-\newline ная из скоростей ветра по румбам за январь, м/с \\

            \\ \hline \endhead
            \hline \subcaption{Продолжение на след. стр.}
            \endfoot
            \hline \endlastfoot
            1 & Абыйский             & п. Белая гора            & -52 & 282 & 21,0 &    &       &       \\ \hline
            2 & Аллаиховский         & п. Чокурдах              & -49 & 318 & 17,5 &    &       &       \\ \hline
            3 & Анабарский           & с. Саскылах              & -60 & -53 & -19,0 & 51 & ЮВ   & 3,4   \\ \hline
            4 & Булунский            & п. Тикси                 & -44 & 365 & 13,4 &    &       & 7,7   \\ \hline
            5 & Верхнеколымский      & п. Зырянка               & -59 & -50 & -20,0 & 82 & С    & 3,0   \\ \hline
            6 & Верхоянский          & п. Верхоянск             & -68 & -58 & -24,7 & 35 & ЮЗ   & 1,4   \\ \hline \clearpage
            7 & Жиганский            & с. Жиганск               & -60 & -52 & -19,8 & 69 & Ю    & 4,1   \\ \hline
            8 & Момский              & с. Хонуу                 & -58 & 275 & 23,5 &    &       &       \\ \hline
            9 & Нижнеколымский       & п. Черский               & -48 & 296 & 16,8 &    &       &       \\ \hline
            10 & Оленекский          & с. Оленек                & -63 & -55 & -18,7 & 67 & В    & 2,6   \\ \hline
            11 & Среднеколымский     & г. Средне-колымск        & -58 & -50 & -19,4 & 72 & ЮЗ   & 2,3   \\ \hline
            12 & Усть-Янский         & п. Депутатский           & -53 & 293 & 21,4 &    &       &       \\ \hline
            13 & Эвено-\newline Бытантайский  & с. Батагай-Алыта         & -55 & 295 & 20,7 &    &       &       \\ 

            \hline
        \end{longtable}
    \end{center}

% \KOMAoptions{paper=seascape}
% \recalctypearea
\end{landscape}


% Таблица 1.3 - Базовые значения требуемого сопротивления теплопередаче ограждающих конструкций для жилых зданий

\begin{landscape}
    
    % \KOMAoptions{paper=landscape}
    % \recalctypearea
        \begin{center}
            \begin{longtable}{|m{5mm}|p{40mm}|p{40mm}|p{30mm}|p{15mm}|p{35mm}|p{35mm}|p{35mm}|p{35mm}|p{35mm}|}
               \caption{Базовые значения требуемого сопротивления теплопередаче ограждающих конструкций для жилых зданий}
                \label{tab:method_gsop_thermalresistvalues}
                \\ \hline
                \multirow{2}{20mm}{№} &
                    \multirow{2}{38mm}{Наименование улуса (района)} &
                    \multirow{2}{38mm}{Улусный (районный) центр} &
                    ГСОП, °С $\times$сут &
                    \multicolumn{4}{c|}{$R_\text{огрконструкции}^\text{требуемое}\text{, м}{^2} \times \text{°С/Вт}$} \\
                \cline{5-8}
                &   &  & &  стен    &
                    покрытий и перекрытий над проездами &
                    Перекрытий чердачных над неотапливаемыми подпольями и подвалами &
                    окон и балконных дверей, витрин и витражей  \\

    
                \hline \endfirsthead
                % \cline{1-10}
                \subcaption{Продолжение таблицы~\ref{tab:method_gsop_thermalresistvalues}}
                \\ \hline \endhead
                \hline \subcaption{Продолжение на след. стр.}
                \endfoot
                \hline \endlastfoot

                1   & Абыйский    & п. Белая гора   & 11844   & 5,55    & 8,12    & 7,23    & 0,80 \\ \hline
                2   & Аллаиховский    & п. Чокурдах & 12243   & 5,69    & 8,32    & 7,41    & 0,81 \\ \hline
                3   & Анабарский  & с. Саскылах & 12280   & 5,70    & 8,34    & 7,43    & 0,81 \\ \hline
                4   & Булунский   & п. Тикси    & 12556   & 5,79    & 8,48    & 7,55    & 0,81 \\ \hline
                5   & Верхнеколымский     & п. Зырянка  & 10906   & 5,22    & 7,65    & 6,81    & 0,77 \\ \hline
                6   & Верхоянский     & п. Верхоянск    & 12430   & 5,75    & 8,42    & 7,49    & 0,81 \\ \hline
                7   & Жиганский   & с. Жиганск  & 11179   & 5,31    & 7,79    & 6,93    & 0,78 \\ \hline
                8   & Момский     & с. Хонуу    & 12238   & 5,68    & 8,32    & 7,41    & 0,81 \\ \hline
                9   & Нижнеколымский  & п. Черский  & 11189   & 5,32    & 7,79    & 6,94    & 0,78 \\ \hline
                10  & Оленекский  & с. Оленек   & 11354   & 5,37    & 7,88    & 7,01    & 0,78 \\ \hline
                11  & Среднеколымский     & г. Средне-колымск   & 11191   & 5,32    & 7,80    & 6,94    & 0,78 \\ \hline
                12  & Усть-Янский & п. Депутатский  & 12423   & 5,75    & 8,41    & 7,49    & 0,81 \\ \hline
                13  & Эвено-Бытантайский  & с. Батагай-Алыта    & 12302   & 5,71    & 8,35    & 7,44    & 0,81 \\ 
    
                \hline
            \end{longtable}
        \end{center}
    
    % \KOMAoptions{paper=seascape}
    % \recalctypearea
\end{landscape}

В условиях повышенной инфильтрации воздуха крайне важным для проектирования зданий является обеспечение
нормируемого значения сопротивления воздухопроницаемости ограждающих конструкций в соответствие с СП 50.13330.2012 \cite{law_RU_Rules_Code_ThermalPerformance}.
Особое внимание следует обратить на воздухонепроницаемость различных стыков ограждающих конструкций, особенно деревянных каркасных домов и зданий из ЛСТК.

Кроме вышеперечисленных требований по тепловой защите зданий в СП 50.13330.2012 \cite{law_RU_Rules_Code_ThermalPerformance}
нормируется значения удельной теплозащитной характеристики здания: % (Таблица 1.3):
% k_об≤k_об^тр .                (3)
\begin{eqndesc}
    \begin{equation}\label{eq:heatprotectequal}
        k_\text{оболочки}   \leqslant   k_\text{оболочки}^\text{требуемая}
    \end{equation}

    % где $   $ -- , \\
    % $   $ -- .
\end{eqndesc}

Удельная теплозащитная характеристика - физическая величина численно равная потерям тепловой энергии единицы отапливаемого объема
в единицу времени при перепаде температуры в 1°С через теплозащитную оболочку здания.
В условиях арктических районов нормируемое значение удельной теплозащитной характеристики $k_\text{оболочки}^\text{требуемая}$  малоэтажных домов составляет:
\begin{itemize}
    \item при отапливаемом объеме 425 м$^3$ (двухквартирный дом с хозблоком в середине общей площадью 152 м$^2$ и высотой 2,8 м) – от 0,298 до 0,324 Вт/(м$^3 \times$°С);
    \item при отапливаемом объеме 851 м$^3$ (четырехквартирный дом с хозблоком в середине общей площадью 304 м$^2$ и высотой 2,8 м) – от 0,237 до 0,257 Вт/(м$^3 \times$°С).
\end{itemize}

Одним из важных показателей для определения класса энергоэффективности жилого дома согласно  СП 50.13330.2012 \cite{law_RU_Rules_Code_ThermalPerformance}
является расчетное значение удельной характеристики расхода тепловой энергии на отопление и вентиляцию здания, которое должно быть меньше или равно нормируемому
значению $q_\text{отоплениерасходэн}^\text{требуемый}$  , Вт/(м$^3 \times$°C):

\begin{eqndesc}
    \begin{equation}\label{eq:heatconsumectequal}
        q_\text{отоплениерасходэн}^\text{расчетный}   \leqslant   q_\text{отоплениерасходэн}^\text{требуемый}
    \end{equation}

    где $q_\text{отоплениерасходэн}^\text{расчетный}$ -- расчетное значение удельной характеристики расхода тепловой энергии на отопление и вентиляцию здания, \\
    $q_\text{отоплениерасходэн}^\text{требуемый}$ -- нормируемая удельная характеристика расхода тепловой энергии на отопление и вентиляцию зданий, Вт/(м$^3 \times$°C),
    определяемая для различных типов жилых и общественных зданий по Таблице 13 или 14 \cite{law_RU_Rules_Code_ThermalPerformance},
    например, для одноэтажного жилого многоквартирного дома равна 0,455 Вт/(м$^3 \times$°C), двухэтажного 0,414 Вт/(м$^3 \times$°C).
\end{eqndesc}

В настоящее время в России используется следующая классификация энергоэффективности зданий:
рейтинг энергоэффективности здания представлен латинскими буквами A++, A+, A, B+, B, C+, C, C-, D, E, где «A++» представляет наивысший рейтинг,
«C» обозначает обычный уровень, а «E» выражает низший уровень.  Данная классификация регламентируется СП 50.13330.2012 Тепловая защита зданий \cite{law_RU_Rules_Code_ThermalPerformance}
и используется при получении энергетического паспорта здания в Российской Федерации. При этом следует отметить, что  согласно п.10.5 \cite{law_RU_Rules_Code_ThermalPerformance},
присвоение зданию класса "В" и "А" производится только при условии включения в проект следующих обязательных энергосберегающих мероприятий:
\begin{itemize}
    \item устройство индивидуальных тепловых пунктов, снижающих затраты энергии на циркуляцию в системах горячего водоснабжения и оснащенных автоматизированными системами управления и учета потребления энергоресурсов, горячей и холодной воды;
    \item применение энергосберегающих систем освещения общедомовых помещений, оснащенных датчиками движения и освещенности;
    \item применение устройств компенсации реактивной мощности двигателей лифтового хозяйства, насосного и вентиляционного оборудования.
\end{itemize}

Кроме в России приняты ряд стандартов, касающихся обеспечения энергетической эффективности зданий.
ГОСТ Р 54862-2011 \Code{[29]} устанавливает метод определения минимальных требований к функциям систем автоматизации
(далее BACS – building automation and control systems) и технического управления зданий (TBM - technical building management),
которые должны внедряться в зданиях различного назначения; методы оценки влияния указанных функций на потребление энергии зданием,
позволяющие ввести характеристики влияния этих функций в расчеты параметров энергетической эффективности зданий.
В данном стандарте подробно приведены функции управления подсистемами: отоплением, вентиляцией и кондиционированием, освещением, системой автоматизации квартир и всего дома, эксплуатацией и технического обслуживания квартир и всего здания в соответствии классам энегоэффективности зданий.
Кроме того, действуют несколько стандартов, оценивающих экономическую эффективность принимаемых в проекте мероприятий.
ГОСТ 56295-2014 \Code{[30]} и ГОСТ 56502-2015 \Code{[31]} устанавливают требования и правила расчетов экономической эффективности
вариантов энергосберегающих мероприятий в зданиях и выбора наиболее целесообразного варианта реализации таких мероприятий.
Таким образом, повышенные требования по тепловой защите зданий обязывает использование энергоэффективных материалов и технологий при строительстве малоэтажных жилых домов
в арктических районах, и ставит задачу для разработки новых конструктивных решений наружных ограждений.
Однозначно, чтобы достичь требуемые значения сопротивления теплопередаче в наружных ограждающих конструкциях, в первую очередь,
необходимо применять современные теплоизоляционные материалы с низким коэффициентом теплопроводности.
Следует отметить, что в арктических районах республики с учетом высоких расходов на отопление зданий необходимо проектировать жилые дома с высоким классом энергосбережения.



