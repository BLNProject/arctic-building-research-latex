\section{Общемировой опыт и парадигма реализации энергоэффективных зданий}


Сокращение добычи природных энергетических ресурсов и, соответственно увеличение их стоимости, с начала XXI века является вызовом для отрасли жилищного строительства в частности и
для проектирования мер развития территорий в целом. Поскольку этот вызов формирует запрос на оптимизацию энергопотребления при развитии территорий во всем мире,
не ограничиваясь арктическими широтами, авторы в данном вводном разделе рассматривают общемировой опыт реализации энергоэффективных зданий в разрезе формирования
общей парадигмы и принципиального подхода к стандартизации процессов энергоэффективного жилого строительства.

Впервые государственную политику в отношении энерго-ресурсосбережения начали проводить в Соединенных Штатах Америки в начале 90-х годов прошлого столетия.
Затем, ряд документов, так называемых конвенций, были приняты в странах ЕС.
Это были в первую очередь государственные законы, стимулирующие внедрение энергосберегающих технологий \Code{[1, 2]}. 

В странах Европы, Канаде и Соединенных Штатах Америки уделяется особенное внимание на энергоэффективность зданий и на методы повышения ее уровня.
В этих странах строят не просто энергоэффективные здания, а так называемые <<зеленые здания>>, к которым регламентируются по западным нормам повышенные требования по безопасности,
негативному воздействию на окружающую среду, также здания должны способствовать комфортному проживанию людей и учитывать интересы будущего поколения \Code{[3]}.
Проектирование экологически устойчивых зданий, <<зеленое>> эко-проектирование (Green Architecture) \Code{[4]} и технологические инновации в данной области - одно из самых популярных и
развиваемых направлений. Канада – занимает одно из первых мест среди них. Так как практически половина потребления всей получаемой от глобальных энергоресурсов энергии
приходится на жилые дома и сооружения, то одним из самых очевидных методов ресурсосбережения становится строительство энергоэффективных и пассивных зданий \Code{[5]}. 
Существенным отличием в порядке проектирования энергоэффективных зданий от нашей страны является тот факт, что на территории Канады и США действует
специальная рейтинговая система сертификации: The Leadership in Energy \& Environmental Design (LEED)\footnote{в переводе <<Лидерство в энергетическом и экологическом проектировании>>} \Code{[6]} .
Система LEED была разработана в 1998 году United States Green Building Council (USGBC) как стандарт измерения проектов энергоэффективных,
экологически чистых и устойчивых зданий для осуществления перехода строительной индустрии к проектированию, строительству и эксплуатации таких зданий.
Важно отметить, что LEED не заменяет собой требования нормативных документов, установленных в той или иной стране государственными ведомствами,
она только дополняет более совершенными, отвечающими запросам современности, критериями оценки качества.

Новый подход помогает решить такие задачи, как снижение уровня потребления энергетических и материальных ресурсов зданием,
снижение неблагоприятного воздействия на природные эко-системы, обеспечение гарантированного уровня комфорта среды обитания человека,
создание новых энергоэффективных и энергосберегающих продуктов, новых рабочих мест в производственном и эксплуатационном секторах,
формирование общественной потребности в новых знаниях и технологиях в области возобновляемой энергетики, формирование у проектировщиков ответственность за
эффективность решений и будущие функции систем \Code{[6]}.

Кроме законодательных требований к энергоэффективности зданий, государствами используется метод поощрения и стимулирования энергосбережения:
\begin{itemize}
    \item Германия ежегодно выделяет субсидии для реконструкции зданий с использованием энергоэффективных технологий;
    \item В Канаде строительство таких зданий всячески поддерживается на уровне правительства: для эко- проектирования предусмотрены льготные условия и финансовые стимулы в законодательстве;
    \item В Швейцарии инвесторы могут вкладывать денежные средства в строительство зданий с низким энергопотреблением и получить грант от государства;
    \item Во Франции жильцы домов, сданных до 1977 г. и решившие утеплить зданий, могут получить скидку и снижение налоговой ставки до 40\% \Code{[7]}.
\end{itemize}


В европейских странах разрабатываются разного рода здания с повышенной энергоэффективностью и пониженным энергопотреблением направленных к уровню <<Zero Energy Building>> \Code{[8]}.
В зарубежном опыте при исследовании зданий на дефекты, ухудшающие тепловую эффективность здания, применяются разного рода тепловизионные устройства в режиме <<time-lapse>>,
при котором заметны все тонкости изменения температурного режима здания после изменения температуры наружной окружающей среды \Code{[9, 10]}.
В работе \Code{[11]} описываются способы и методы интеграции энергоемких материалов в строительных конструкциях.
Сам факт использования таких материалов имеет хороший потенциал для зданий при сокращении энергопотребления, но требует некоторых инвестиций во время строительства. Инвестирование и затраты на повышение тепловой эффективности здания выгодны по экологическим и почти по всем экономическим критериям \Code{[12]}. В данном случае описываются затраты на потребление 1 кВт*ч в здании, при котором у здания со слабой тепловой защитой повышаются затраты на потребление энергетических ресурсов. 
В Китае нормативом, использующимся для оценки гражданских <<зеленых>> зданий, является <<GB-T 50378-2019 Стандарт оценки зеленых зданий>>,
который является переизданным изданием аналогичного норматива 2014 года с повышением требований.
Система индексов оценки <<зеленых>> зданий включает пять типов показателей: безопасность и долговечность, здоровье и комфорт, удобство проживания,
уровень ресурсосбережения и комфортность окружающей среды. Каждый показатель имеет обязательные и дополнительные требования для получения баллов \Code{[13]}.

% Перенос в методы регулирования РФ


