\subsection{\scAssesmentSystem{BREEAM}}
BREEAM~--- \mbox{англ.}~Building  Research Establishment Environmental Assessment Method), рейтинговая  система  оценки «зеленых» зданий, разработанная в 1990~г.
на базе Строительного научно-исследовательского института Великобритании (Building Research Establishment, BRE) и контролируемая организацией BRE Global.


% Юрисдикция
\subsubsection*{\scAssesmentScope}
Систему допускается применять на любой территории при условии выполнения критериев оценки.
BRE ведёт свою деятельность преимущественно на территории Совединенного Королевства, стран Содружества и Евросоюза.

BREEAM предполагает добровольное применение и сертификацию в формате сопровождения проекта ответственным лицом, назначаемым организацией-оператором BREEAM.
Как правило, это один человек, а не группа экспертов \cite{method_GB_BREEAM}.

% Метрики
\subsubsection*{Описание оцениваемых показателей и метрик оценки зданий}
Метрики оценки группируются по смежным предметным областям на основе видения (стратегических положений, целеполагания), которые формируются по следующим направлениям строительной деятельности:
\begin{enumerate}[1)]
    \item Net zero carbon (Нулевые выбросы углерода) --- видение заключается в построении искусственной среды, отвечающей требованиям чрезвычайной климатической ситуации.
        BREEAM поддерживает нынешнее и будущие поколения в создании устойчивой среды, отвечающей его экологическим и коммерческим целям.
        BREEAM представляет искусственную среду как инструмент в достижении нулевого уровня выбросов во всем мире к 2050 году и недопущении глобального потепления на 1,5 градуса,
        BREEAM поддерживает решения по обезуглероживанию застроенной среды, недвижимости и связанных с ними инвестиций за счет 
        минимизации выбросов углерода при разработке, реконструкции и эксплуатации активов, предоставления методологий оценки выбросов углерода,
        поощрения использования возобновляемых источников энергии на площадке объекта строительства и предоставление кредитов на энергию и сокращение выбросов углерода,
        обеспечения сторонней проверки оценки выбросов углерода;
    \item Whole life performance (WLP-подход к разработке) --- парадигма, в рамках которой рассматривается воздействие здания (актива, продукции) на окружающую среду в рамках жизненного цикла целиком ---
        от проекта до утилизации (т.е. сноса);
        Команды проектировщиков больше не работают изолированно от организаций-эксплуатантов (операторов) активов --- недвижимого имущества, и отрасль быстро движется к подходу,
        основанному на производительности на протяжении всего срока службы (WLP);
    
        С учетом этого были созданы технические стандарты BREEAM. Подход заключается в том, чтобы постоянно принимать лучшие решения
        на протяжении всего жизненного цикла здания посредством сертификации по стандартам BREEAM.
        BREEAM предоставляет инструменты и рамки для расширения возможностей инвесторов, разработчиков,
        владельцам и операторам принимать эти обоснованные решения в течение всего срока службы их активов, а также измерять и сообщать об этих результатах;
        
        BREEAM поддерживает производительность на протяжении всего срока службы за счет содействия процессам вторичной переработки материалов,
        оборудования, механизмов в составе активов (зданий и другого недвижимого имущества) на протяжении всего жизненного цикла среды объекта капитального строительства,
        обеспечения целостного подхода к оценке устойчивости с учетом экологических, социальных и экономических последствий,
        предоставления научной основы для балансировки различных целей и задач,
        помоощи в выявлении разрывов в производительности между проектным замыслом и эксплуатационными характеристиками посредством аналитики данных,
        с целью поддерживать постоянное улучшение показателей недвижимости;
    \item Health \& social impact (Влияние на здоровье и на социум) --- BREEAM поддерживает решения в области здравоохранения и социального воздействия за счет
        предоставления научно обоснованных методов оценки здоровья, благополучия и социального воздействия на всех этапах жизненного цикла,
        предоставления методов сбора медицинских и социальных данных, управление и проверки, предоставления сторонней гарантии результатов;
    \item Circularity \& resilience (Цикличность и устойчивость) --- BREEAM предлагает решения в области цикличности (повторного использования материалов и активов) и устойчивости за счет
        поощрения конструктивных решений для обеспечения прочности и увеличения срока службы актива (здания), обеспечения ответственного выбора материалов,
        снижения потребления воды и энергии, предоставление методов сокращения отходов;
    \item Biodiversity (Биоразнообразие) --- включает разработку стратегий восстановления и защиты биоразнообразия,
        согласованные и преемственные решения с инициативами международных экологических и других организаций (т.к. SDG \cite{UN_17Goals}),
        с целью помощи в интеграции применения принципов защиты природы высокого уровня;
    \item Disclosures \& reporting (Раскрытие информации и отчетность) --- упрощение раскрытия информации и отчетности с помощью BREEAM, а именно:
        поддержка отчетов и решений ESG\footnote{Environmental, Social, and Governance (экологическое, социальное и управленческое) \cite{business_US_ESG},
            Парадигма оценки инвестиционных вложений во взаимоувязке трех аспектов хозяйственной деятельности предприятия, впервые появившаяся в 2005 году
            и публикующаяся как оценка по показателям в отчетах коммерческих организаций, деятельность которых публична;
            ESG охватывает широкий круг вопросов, которые могут прямо или косвенно влиять на финансовую значимость.
            Некоторые из этих вопросов, которые входят в компетенцию отчетности ESG, включают управление ресурсами, управление цепочками поставок,
            здоровье организации, политику безопасности, и укрепление доверия посредством прозрачности.},
        информирование и обмен знаниями между классификаторами Евросоюза и Великобритании,
        сопоставление схем BREEAM с Целями ООН в области устойчивого развития (ЦУР) \cite{UN_17Goals},
        предоставление механизма для получения <<зеленого>> финансирования и измерения успеха,
        предоставление сторонней проверки и подтверждения заявлений об устойчивости;
    \item EU Taxonomy (Классификаторы Евросоюза) --- представляет собой четкую и подробную систему классификации, используемую для определения экологически устойчивой экономической деятельности;
        Система включает шесть целей, известных как <<приложения>>, в соответствии с которыми, предприятия должны продемонстрировать, что они вносят существенный вклад в достижение цели,
        при соблюдении критериев <<не причинить значительного вреда>> для остальных пяти целей;
        Первые две из этих целей~--- смягчение последствий изменения климата и адаптация к его изменению~--- были подтверждены Европейской комиссией,
        а третья, касающаяся циркулярности (вторичного использования сырья), должна быть завершена летом 2022 года;                                
        BRE обязалась действовать в соответствии с Классификаторами Евросоюза в BREEAM, для достижения их клиентами целей в области устойчивого развития, что, согласно позиции BRE,
            обеспечит прочную основу для соблюдения требований, роста «зеленых» облигаций и возможностей устойчивого финансирования для зданий BREEAM.
\end{enumerate}

% Модель
\subsubsection*{Модель и способы оценки}
Модель оценки выстраивается на принципах сопровождения процесса разработки проекта ответственным лицом, именуемым appraiser (<<оценщик>>).
При разработке проекта (обычно с привлечением консультанта по BREEAM) его проверяет оценщик — один человек, а не группа экспертов.
Оценщик действует очень формально: ему предоставляются документы для проверки соответствия установленным критериям, и, в случае соответствия, он их одобряет.
Оценщик появляется дважды – на стадии проектирования и на стадии сдачи объекта в эксплуатацию.
По результатам проверки соответствия он готовит отчет, который затем отправляется в Великобританию для выборочной проверки. По результатам оценки выдается сертификат.
Важно отметить, что в системе BREEAM сертификат выдается дважды (за сам проект и при его завершении), сертификат LEED выдается один раз после завершения объекта.

Еще одно важное различие между этими двумя системами заключается в том, что LEED основывает свои пороговые значения на процентах, тогда как BREEAM опирается на количественные стандарты.
Также нюанс системы LEED в том, что есть очень строгие обязательные требования, которые необходимы для оформления — иначе здание не будет считаться зеленым.
BREEAM как система оценки более гибкая за счет <<прогрессивного подхода>>: обязательные базовые требования смягчены относительно LEED и чем выше уровень сертификации, тем выше требования.
% Например, на минимальном уровне одно из обязательных условий: чтобы люминесцентные лампы были высокочастотными и не мерцали.
% Это очень просто, ведь все современное оборудование не мерцает.

Техническое обеспечение оценки осуществляется с применением цифровых удаленных служб (web-servicies) на базе платформ с графическими пользовательскими оболочками и средствами удаленного обмена данными.
На период исследования реализован сетевой ресурс https://www.greenbooklive.com, агрегирующий стандарты и цифровые инструменты оценки строительства <<зеленых>> зданий,
RESTful API веб-служба\footnote{доступ предоставляется по запросу из формы https://bregroup.com/products/breeam/breeam-tools/breeam-api/},
включающая сертифицированные оценки проектов или активов по схемам BREEAM, HQM и CEEQUAL, к которой производится обращение на стадии
технико-экономического обоснования строительства с целью получения аналитики опыта реализации <<зеленых>> зданий и принятия решений на основе данных (data-driven economy),
размещенных в едином реляционном хранилище и описывающих этот опыт.
Весь процесс сертификации сопровождается единым ответственным лицом --- <<оценщиком>>, связанным с аффилированной организацией (BREEAM National Scheme Operator, NSO).
Оценка производится экспертным методом на основе технических стандартов --- BREEAM Schemas (схемы BREEAM), имеющих цифровое воплощение в виде машиночитаемых множеств данных,
применяемых для автоматизации сопоставления критериев BREEAM по актуальным исследовательским данным и моделям оценки в текущей версии стандарта.
Система позиционируется как интегрируемая в процессы капиталистической модели, применяемой при реализации строительно-инвестиционных проектов,
и базируется на экономической деловой модели SaaS\footnote{англ. software as a service — программное обеспечение как услуга; также англ. software on demand — программное обеспечение по требованию;
    в случае с BREEAM предоставляется доступ к программному обеспечению, сопровождающему процесс оценки и к экспертному сообществу, ведущему консультационные,
    расчетные и иные услуги, которые востребованны в актуальной редакции системы BREEAM},
при которой поставщик (оператор BREEAM) берет на себя ряд обязанностей по инфраструктурному обеспечению процесса  рейтинга <<зеленых>> зданий.
Таким образом, модели оценки~--- как математические, так и экономические,~--- обновляются совместно с инфраструктурой и, в частности, с программным обеспечением по Agile (гибкой) методологии \cite{method_GB_BREEAM}.
