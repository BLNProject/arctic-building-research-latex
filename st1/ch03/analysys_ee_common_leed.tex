\subsection{\scAssesmentSystem{LEED}}
LEED (\mbox{англ.}~The Leadership in Energy \& Environmental Design)\footnote{рус. <<Лидерство в Энергетическом и Средовом Проектировании>>} \cite{method_US_LEED}:
рейтинговая система оценки «зеленых» зданий, разработанная в 1998~году United States Green Building Council (USGBC)\footnote{рус. Совет по Зеленому Строительству Соединенных Штатов}
как стандарт измерения проектов энергоэффективных, экологически чистых зданий для осуществления перехода строительной индустрии к неразрывно целостному процессу
проектирования, строительства и эксплуатации таких зданий.
Важно отметить, что LEED не заменяет собой требования нормативных документов, установленных в той или иной стране государственными ведомствами,
она только дополняет более совершенными, отвечающими запросам современности, критериями оценки качества.



% Юрисдикция
\subsubsection*{\scAssesmentScope}
Сертификацию применяют преимущественно Соединенные Штаты Америки и государства-союзники США. Изначально LEED-сертификация не имеет территориальных ограничений,
однако USGBC является организацией, находящейся в правовом поле и юрисдикции США.
% С января 2022-го ратифицированная версия действует и в РФ
Среди арктических государств, в дополнении к США, система LEED применяется к зданиям на территории Канады.

Стандарт применяется добровольно или по требованиям Задания на проектирования. В этом случае в проектную команду включается аккредитованный
LEED-специалист в роли проектировщика (как правило -- архитектора) или руководителя проектной команды.
При этом итоговый проект по завершении разработки проходит техническую экспертизу в USGBC, проводимую группой экспертов.

% Метрики
\subsubsection*{Описание оцениваемых показателей и метрик оценки зданий}
Подход LEED преемствует идеи BREEAM и помогает решить такие задачи, как снижение уровня потребления энергетических и материальных ресурсов зданием,
снижение неблагоприятного воздействия на природные эко-системы, обеспечение гарантированного уровня комфорта среды обитания человека,
создание новых энергоэффективных и энергосберегающих продуктов, новых рабочих мест в производственном и эксплуатационном секторах,
формирование общественной потребности в новых знаниях и технологиях в области возобновляемой энергетики, формирование у проектировщиков ответственности за
эффективность решений и будущие функции систем \cite{method_US_LEED}.

Система сертификации экологически чистых зданий, обеспечивает стороннюю проверку того, что конкретное здание или комплекс были спроектированы и построены с учетом следующих параметров:
\begin{enumerate}[1)]
    \item Максимальная экономия энергии;
    \item Эффективное использование воды;
    \item Снижение выбросов парниковых газов;
    \item Более здоровое качество воздуха в помещении;
    \item Более широкое использование переработанных материалов;
    \item Оптимальное использование ресурсов и чувствительность к их воздействию;
    \item Снижение затрат на техническое обслуживание и эксплуатацию.
\end{enumerate}

% Модель
\subsubsection*{Модель и способы оценки}
Сертификация здания по LEED-системе производится методом экспертной оценки группой технической экспертизы на предмет соответствия критериям после подготовки комплекта документации и выполнения формальных требований в USGBC.
В бально-рейтинговой системе выделяются итоговые рейтинговые позиции~--- Certified (Проверено), Silver (Серебро), Gold (Золото), Platinum (Платина), т.н. <<уровень сертификации>> (Certification level)~---
для достижения которых требуется набор установленного количества Points (рейтинговых баллов).

Каждый параметр содержит связанный набор критериев, которые имеют статус обязательных и дополнительных.
Каждый набор критериев в составе параметров статичен и имеет строгие требования: в случае несоответствия обязательным требованиям на предыдущих рейтинговых позициях,
проект не имеет возможности получить следующий рейтинг. Например, для получения Gold необходимо, чтобы базовые критерии рейтингов Certified и Silver были реализованы в проекте.
Пороговые значения для достижения соответствия основываются на долевых пропорциях и коэффициентах (в процентах), а не на количественных показателях объекта.
Дополнительные критерии применяются и начисление добавочных Points по ним производится только после подтверждения соблюдения обязательных критериев для каждого из параметров \cite{method_US_LEED}.
