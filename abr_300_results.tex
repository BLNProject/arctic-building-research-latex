\chapter{Обзор и оценка градостроительных принципов и способов реализации жилищного строительства в условиях Арктики}
В главе авторы рассматривают способы выживания и расселения, сформированные коренным населением Арктики.
Производится обощение опыта и выявление наиболее востребованных практик, применимых при градостроительном планировании и принятии решений,
на основе сравнения опыта организации расселения коренного населения с современными практиками управления территориями, на следующих уровнях:
\begin{enumerate}[1)]
    \item государственном (федеральном) — уровень принятия решений высшего руководства страны и подведомственных правительству организации — министерств, ведомств, надзорных органов и др.;
    \item региональном (субъекта) — уровень принятия решений на территории государства, для которой законодательством этого государства определены принципы организации самоуправления;
    \item общинном — уровень принятия решений сообществ, товариществ собственников, гильдий и аналогичных схожих по организационной форме
    социальных образований, действующих на основании устава или общей общественной идеи;
    \item частного домохозяйства (частном) — уровень принятия решения отдельной ячейки общества — семьи, индивида и др., распоряжающихся частным
    домохозяйством, собственными материальными ресурсами и капиталом.
\end{enumerate}


% Сниски \footnote[8]{Восьмая}
% или вусё же сноски \footnote[]{Восьмая}

