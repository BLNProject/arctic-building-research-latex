% \clearpage

% \begin{landscape}
    
% % \KOMAoptions{paper=landscape}
% % \recalctypearea
%     \begin{center}
%         \begin{longtable}{|m{0.02\textwidth}|p{0.2\textwidth}|p{0.2\textwidth}|p{0.12\textwidth}|p{0.13\textwidth}|p{0.12\textwidth}|p{0.14\textwidth}|p{0.14\textwidth}|p{0.14\textwidth}|p{0.14\textwidth}|}
%            \caption{Расчетные параметры температуры по арктическим улусам РС(Я)}
%             \label{tab:pethod_gsop_calcvalues}
%             \\ \hline
%                 \multirow{2}{18cm}{№} &
%                 \multirow{2}{18cm}{Наименование\\ улуса\\ (района)} &
%                 \multirow{2}{18cm}{Улусный \\ (районный)\\ центр} &
%                 \multirow{2}{18cm}{Абсолютно\\ минималь- ная температура\\ воздуха, °С} &
%                 \multirow{2}{18cm}{Температура\\ воздуха\\ наиболее\\ холодной\\ пятидневки,\\ °С, с обес-\\ печен- ностью 0,92} &
%                 Продолжи- тельность, сут, и средняя температура воздуха, °С, периода со средней суточной температурой воздуха $\leqslant$ 8°С &       &
%                 \multirow{2}{18cm}{Количество\\ осадков\\ ноябрь-март,\\ мм} &
%                 \multirow{2}{18cm}{Преоблада-\\ ющее\\ направление ветра\\ декабрь-февраль} &
%                 \multirow{2}{18cm}{Максималь-\\ ная из скоростей\\ ветра по румбам\\ за январь, м/с} \\
            
                
%             \cline{6-7}
%              & & & & & Продолжи- тельность & Средняя температура  & &  \\

%             \hline \endfirsthead
%             % \cline{1-10}
%             \subcaption{Продолжение таблицы~\ref{tab:method_gsop_calcvalues}}
%             \\ \hline \endhead
%             \hline \subcaption{Продолжение на след. стр.}
%             \endfoot
%             \hline \endlastfoot
%             1 & Абыйский             & п. Белая гора            & -52 & 282 & 21,0 &    &       &       \\ \hline
%             2 & Аллаиховский         & п. Чокурдах              & -49 & 318 & 17,5 &    &       &       \\ \hline
%             3 & Анабарский           & с. Саскылах              & -60 & -53 & -19,0 & 51 & ЮВ   & 3,4   \\ \hline
%             4 & Булунский            & п. Тикси                 & -44 & 365 & 13,4 &    &       & 7,7   \\ \hline
%             5 & Верхнеколымский      & п. Зырянка               & -59 & -50 & -20,0 & 82 & С    & 3,0   \\ \hline
%             6 & Верхоянский          & п. Верхоянск             & -68 & -58 & -24,7 & 35 & ЮЗ   & 1,4   \\ \hline
%             7 & Жиганский            & с. Жиганск               & -60 & -52 & -19,8 & 69 & Ю    & 4,1   \\ \hline
%             8 & Момский              & с. Хонуу                 & -58 & 275 & 23,5 &    &       &       \\ \hline
%             9 & Нижнеколымский       & п. Черский               & -48 & 296 & 16,8 &    &       &       \\ \hline
%             10 & Оленекский          & с. Оленек                & -63 & -55 & -18,7 & 67 & В    & 2,6   \\ \hline
%             11 & Среднеколымский     & г. Средне-колымск        & -58 & -50 & -19,4 & 72 & ЮЗ   & 2,3   \\ \hline
%             12 & Усть-Янский         & п. Депутатский           & -53 & 293 & 21,4 &    &       &       \\ \hline
%             13 & Эвено-Бытантайский  & с. Батагай-Алыта         & -55 & 295 & 20,7 &    &       &       \\ 

%             \hline
%         \end{longtable}
%     \end{center}

% % \KOMAoptions{paper=seascape}
% % \recalctypearea
% \end{landscape}


\begin{landscape}
    
% \KOMAoptions{paper=landscape}
% \recalctypearea
    \begin{center}
        \begin{longtable}{|m{5mm}|p{35mm}|p{30mm}|p{22mm}|p{22mm}|p{22mm}|p{22mm}|p{22mm}|p{22mm}|p{22mm}|}
           \caption{Расчетные параметры температуры по арктическим улусам РС(Я)}
            \label{tab:method_gsop_calcvalues}
            \\ \hline
            № &
                Наименование улуса (района)&
                Улусный (районный) центр &
                Абсолютно минимальная температура воздуха, °С &
                Температура воздуха наиболее холодной пятидневки, °С, с обеспеченностью 0,92 &
                Продолжи-\newline тельность, сут, периода со средней суточной температурой воздуха $\leqslant$8°С &
                Cредняя температура воздуха, °С, периода со средней суточной температурой воздуха $\leqslant$8°С &
                Количество осадков ноябрь-март, мм &
                Преоблада-\newline ющее направление ветра декабрь-февраль &
                Максималь-\newline ная из скоростей ветра по румбам за январь, м/с \\

            \hline \endfirsthead
            % \cline{1-10}
            \subcaption{Продолжение таблицы~\ref{tab:method_gsop_calcvalues}}

            \\ \hline
            № &
                Наименование улуса (района)&
                Улусный (районный) центр &
                Абсолютно минимальная температура воздуха, °С &
                Температура воздуха наиболее холодной пятидневки, °С, с обеспеченностью 0,92 &
                Продолжи-\newline тельность, сут, периода со средней суточной температурой воздуха $\leqslant$8°С &
                Cредняя температура воздуха, °С, периода со средней суточной температурой воздуха $\leqslant$8°С &
                Количество осадков ноябрь-март, мм &
                Преоблада-\newline ющее направление ветра декабрь-февраль &
                Максималь-\newline ная из скоростей ветра по румбам за январь, м/с \\

            \\ \hline \endhead
            \hline \subcaption{Продолжение на след. стр.}
            \endfoot
            \hline \endlastfoot
            1 & Абыйский             & п. Белая гора            & -52 & 282 & 21,0 &    &       &       \\ \hline
            2 & Аллаиховский         & п. Чокурдах              & -49 & 318 & 17,5 &    &       &       \\ \hline
            3 & Анабарский           & с. Саскылах              & -60 & -53 & -19,0 & 51 & ЮВ   & 3,4   \\ \hline
            4 & Булунский            & п. Тикси                 & -44 & 365 & 13,4 &    &       & 7,7   \\ \hline
            5 & Верхнеколымский      & п. Зырянка               & -59 & -50 & -20,0 & 82 & С    & 3,0   \\ \hline
            6 & Верхоянский          & п. Верхоянск             & -68 & -58 & -24,7 & 35 & ЮЗ   & 1,4   \\ \hline \clearpage
            7 & Жиганский            & с. Жиганск               & -60 & -52 & -19,8 & 69 & Ю    & 4,1   \\ \hline
            8 & Момский              & с. Хонуу                 & -58 & 275 & 23,5 &    &       &       \\ \hline
            9 & Нижнеколымский       & п. Черский               & -48 & 296 & 16,8 &    &       &       \\ \hline
            10 & Оленекский          & с. Оленек                & -63 & -55 & -18,7 & 67 & В    & 2,6   \\ \hline
            11 & Среднеколымский     & г. Средне-колымск        & -58 & -50 & -19,4 & 72 & ЮЗ   & 2,3   \\ \hline
            12 & Усть-Янский         & п. Депутатский           & -53 & 293 & 21,4 &    &       &       \\ \hline
            13 & Эвено-\newline Бытантайский  & с. Батагай-Алыта         & -55 & 295 & 20,7 &    &       &       \\ 

            \hline
        \end{longtable}
    \end{center}

% \KOMAoptions{paper=seascape}
% \recalctypearea
\end{landscape}
