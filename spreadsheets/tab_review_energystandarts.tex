\begin{landscape}
    
    % \KOMAoptions{paper=landscape}
    % \recalctypearea
        \begin{center}
            \begin{longtable}{|m{40mm}|p{40mm}|p{40mm}|p{55mm}|p{55mm}|}
               \caption{Общемировые стандарты и системы рейтинговой оценки энергетической эффективности зданий}
                \label{tab:review_energystandarts}
                \\ \hline
                % \multirow{2}{20mm}{№} &
                %     \multirow{2}{38mm}{Наименование улуса (района)} &
                %     \multirow{2}{38mm}{Улусный (районный) центр} &
                %     ГСОП, °С $\times$сут &
                %     \multicolumn{4}{c|}{$R_\text{огрконструкции}^\text{требуемое}\text{, м}{^2} \times \text{°С/Вт}$} \\
                % \cline{5-8}
                % &   &  & &  стен    &
                %     покрытий и перекрытий над проездами &
                %     Перекрытий чердачных над неотапливаемыми подпольями и подвалами &
                %     окон и балконных дверей, витрин и витражей  \\
                Наименование \mbox{стандарта} & Юрисдикция (страны действия) & Способ применения (добровольно/как часть НПА) &
                    Описание метрик (что оценивается) & Описание модели (как оценивается) \\

    
                \hline \endfirsthead
                % \cline{1-10}
                \subcaption{Продолжение таблицы~\ref{tab:review_energystandarts}}
                \\ \hline \endhead
                \hline \subcaption{Продолжение на след. стр.}
                \endfoot
                \hline \endlastfoot

                        % &       &   &    &
                        % Совединенное Королевство и страны Содружества, США,
                BREEAM (\mbox{англ.}~Building  Research Establishment Environmental Assessment Method): рейтинговая  система  оценки «зеленых» зданий,разработанная в 1990~г. британской организацией BRE Global. &
                    Возможно применять на любой территории при условии выполнения критериев оценки. BRE ведёт свою деятельность преимущественно на территории Совединенного Королевства, страны Содружества и Евросоюза &
                    Добровольное применение и сертификация. Осуществляется с применением цифровых удаленных служб (web-servicies) на базе платформ с графическими пользовательскими оболочками и средствами удаленного обмена данными.
                    На период исследования реализован сетевой ресурс https://www.greenbooklive.com, агрегирующий стандарты и цифровые инструменты оценки строительства <<зеленых>> зданий,
                    RESTful API веб-служба \footnote{доступ предоставляется по запросу из формы https://bregroup.com/products/breeam/breeam-tools/breeam-api/},
                    включающая сертифицированные оценки проектов или активов по схемам BREEAM, HQM и CEEQUAL, к которой производится обращение на стадии
                    технико-экономического обоснования строительства с целью получения аналитики опыта реализации <<зеленых>> зданий и принятия решений на основе данных (data-driven economy),
                    размещенных в едином реляционном хранилище и описывающих этот опыт. Весь процесс сертификации сопровождается единым ответственным лицом --- <<оценщиком>>, связанным с аффилированной организацией (BREEAM National Scheme Operator, NSO) &
                    Метрики оценки отражаются в веб-службах, которые формируются по следующим направлениям строительной деятельности:
                        \begin{enumerate}[1)]
                            \item Net zero carbon (Нулевые выбросы углерода) --- видение заключается в построении искусственной среды, отвечающей требованиям чрезвычайной климатической ситуации.
                                BREEAM поддерживает нынешнее и будущие поколения в создании устойчивой среды, отвечающей его экологическим и коммерческим целям.
                                BREEAM представляет искусственную среду как инструмент в достижении нулевого уровня выбросов во всем мире к 2050 году и недопущении глобального потепления на 1,5 градуса,
                                BREEAM поддерживает решения по обезуглероживанию застроенной среды, недвижимости и связанных с ними инвестиций за счет 
                                минимизации выбросов углерода при разработке, реконструкции и эксплуатации активов, предоставления методологий оценки выбросов углерода,
                                поощрения использования возобновляемых источников энергии на площадке объекта строительства и предоставление кредитов на энергию и сокращение выбросов углерода,
                                Обеспечение сторонней проверки оценки выбросов углерода;
                            \item Whole life performance (WLP) --- парадигма, в рамках которой рассматривается полное воздействие, которое здание может оказать на окружающую среду.
                                Команды проектировщиков больше не работают изолированно от организаций-эксплуатантов (операторов) активов --- недвижимого имущества, и отрасль быстро движется к подходу,
                                основанному на производительности на протяжении всего срока службы (WLP).
                            
                                С учетом этого были созданы технические стандарты BREEAM. Подход заключается в том, чтобы постоянно принимать лучшие решения
                                на протяжении всего жизненного цикла здания посредством сертификации по стандартам BREEAM.
                                BREEAM предоставляет инструменты и рамки для расширения возможностей инвесторов, разработчиков,
                                владельцам и операторам принимать эти обоснованные решения в течение всего срока службы их активов, а также измерять и сообщать об этих результатах
                                
                                BREEAM поддерживает производительность на протяжении всего срока службы за счет содействия процессам вторичной переработки материалов,
                                оборудования, механизмов в составе активов (зданий и другого недвижимого имущества) на протяжении всего жизненного цикла среды объекта капитального строительства,
                                обеспечения целостного подхода к оценке устойчивости с учетом экологических, социальных и экономических последствий,
                                предоставления научной основы для балансировки различных целей и задач,
                                помоощи в выявлении разрывов в производительности между проектным замыслом и эксплуатационными характеристиками посредством аналитики данных,
                                с целью поддерживать постоянное улучшение показателей недвижимости;
                            \item Health \& social impact (Влияние на здоровье и на социум) --- BREEAM поддерживает решения в области здравоохранения и социального воздействия за счет
                                предоставления научно обоснованных методов оценки здоровья, благополучия и социального воздействия на всех этапах жизненного цикла,
                                предоставления методов сбора медицинских и социальных данных, управление и проверки, предоставления сторонней гарантии результатов;
                            \item Circularity \& resilience (Цикличность и устойчивость) --- BREEAM предлагает решения в области цикличности (повторного использования материалов и активов) и устойчивости за счет
                                поощрения конструктивных решений для обеспечения прочности и увеличения срока службы актива (здания), обеспечения ответственного выбора материалов,
                                снижения потребления воды и энергии, предоставление методов сокращения отходов;
                            \item Biodiversity (Биоразнообразие) --- включает разработку стратегий восстановления и защиты биоразнообразия,
                                согласованные и преемственные решения с инициативами международных экологических и других организаций (т.к. SDG \cite{UN_17Goals}),
                                с целью помощи в интеграции применения принципов защиты природы высокого уровня;
                            \item Disclosures \& reporting (Раскрытие информации и отчетность) --- упрощение раскрытия информации и отчетности с помощью BREEAM, а именно:
                                поддержка отчетов и решений ESG\footnote{Environmental, Social, and Governance (экологическое, социальное и управленческое) \cite{business_US_ESG},
                                    Парадигма оценки инвестиционных вложений во взаимоувязке трех аспектов хозяйственной деятельности предприятия, впервые появившаяся в 2005 году
                                    и публикующаяся как оценка по показателям в отчетах коммерческих организаций, деятельность которых публична;
                                    ESG охватывает широкий круг вопросов, которые могут прямо или косвенно влиять на финансовую значимость.
                                    Некоторые из этих вопросов, которые входят в компетенцию отчетности ESG, включают управление ресурсами, управление цепочками поставок,
                                    здоровье организации, политику безопасности, и укрепление доверия посредством прозрачности.},
                                информирование и обмен знаниями между классификаторами Евросоюза и Великобритании,
                                сопоставление схем BREEAM с Целями ООН в области устойчивого развития (ЦУР) \cite{UN_17Goals},
                                предоставление механизма для получения <<зеленого>> финансирования и измерения успеха,
                                предоставление сторонней проверки и подтверждения заявлений об устойчивости;
                            \item EU Taxonomy (Классификаторы Евросоюза) --- представляет собой четкую и подробную систему классификации, используемую для определения экологически устойчивой экономической деятельности;
                                Система включает шесть целей, известных как <<приложения>>, в соответствии с которыми, предприятия должны продемонстрировать, что они вносят существенный вклад в достижение цели,
                                при соблюдении критериев <<не причинить значительного вреда>> для остальных пяти целей;
                                Первые две из этих целей~--- смягчение последствий изменения климата и адаптация к его изменению~--- были подтверждены Европейской комиссией,
                                а третья, касающаяся циркулярности (вторичного использования сырья), должна быть завершена летом 2022 года;                                
                                BRE обязалась действовать в соответствии с Классификаторами Евросоюза в BREEAM, для достижения их клиентами целей в области устойчивого развития, что, согласно позиции BRE,
                                 обеспечит прочную основу для соблюдения требований, роста «зеленых» облигаций и возможностей устойчивого финансирования для зданий BREEAM.
                        \end{enumerate} &
                    Оценка производится экспертным методом. Система позиционируется как интегрируемая в процессы капиталистической модели, применяемой при реализации строительно-инвестиционных проектов,
                    и базируется на экономической деловой модели SaaS\footnote{англ. software as a service — программное обеспечение как услуга; также англ. software on demand — программное обеспечение по требованию;
                        в случае с BREEAM предоставляется доступ к программному обеспечению, сопровождающему процесс оценки и к экспертному сообществу, ведущему консультационные,
                        расчетные и иные услуги, которые востребованны в актуальной редакции системы BREEAM},
                    при которой поставщик (оператор BREEAM) берет на себя ряд обязанностей по инфраструктурному обеспечению процесса  рейтинга <<зеленых>> зданий.
                    Таким образом, модели оценки~--- как математические, так и экономические,~--- обновляются совместно с инфраструктурой и, в частности, с программным обеспечением по Agile (гибкой) методологии
                     \\ \hline
                LEED (\mbox{англ.}~The Leadership in Energy \& Environmental Design): рейтинговая система оценки «зеленых» зданий, разработанная в 1998~г. Американским советом USGBC. &
                    Соединенные Штаты Америки и государства-союзники США. Изначально LEED-сертификация не имеет территориальных ограничений, однако USGBC является организацией, находящейся в правовом поле и юрисдикции США &
                    Стандарт применяется добровольно или по требованиям Задания на проектирования. В этом случае в проектную команду включается аккредитованный
                    LEED-специалист в роли проектировщика (как правило -- архитектора) или руководителя проектной команды &
                    Система сертификации экологически чистых зданий, обеспечивает стороннюю проверку того, что конкретное здание или комплекс были спроектированы и построены с учетом следующих параметров:
                    \begin{enumerate}[1)]
                        \item Максимальная экономия энергии;
                        \item Эффективное использование воды;
                        \item Снижение выбросов парниковых газов;
                        \item Более здоровое качество воздуха в помещении;
                        \item Более широкое использование переработанных материалов;
                        \item Оптимальное использование ресурсов и чувствительность к их воздействию;
                        \item Снижение затрат на техническое обслуживание и эксплуатацию.
                    \end{enumerate}
                    &
                    Сертификация по LEED здания производится методом экспертной оценки на предмет соответствия критериям после подготовки комплекта документации и выполнения формальных требований в USGBC.
                    \\ \hline
                DGNB (\mbox{нем.}~Deutsche Gesellschaft fur Nachhaltiges Bauen): рейтинговая система оценки «зеленых» зданий, разработанная в 2007~г. Немецким советом по устойчивому строительству. &
                    Действует на территории стран Евросоюза &
                    Система DGNB оценивает не отдельные показатели, а общую производительность здания в ходе жизненного цикла на основе критериев.
                    Если эти критерии выполняются превосходно, здание получает сертификат или предварительный сертификат (платина, золото, серебро или бронза) для существующей недвижимости.
                    DGNB продолжает развивать свою систему сертификации и адаптирует ее к национальным и международным стандартам и законодательству,
                    и, преемствуя парадигму системы Eurocode, носит рекомендательный характер &
                    Оценка производится по критериям, которые синхронизированны с 17-ю целями устойчивого развития ООН \cite{UN_17Goals} и техническими регламентами среды обитания человека,
                    закреплёнными в межгосударственных стандартах Eurocode. Критерии выделяются в следующие группы --- в соответствии с отраслями жизнедеятельности человека и предметными областями,
                    задействованными в развитии территорий:
                    \begin{enumerate}[1)]
                        \item Environmental quality (Качество среды) --- шесть критериев качества окружающей среды позволяют проводить оценку воздействия зданий на глобальную и местную окружающую среду, а также воздействия на ресурсы и образования отходов;
                        \item Economic quality (Качество хозяйственной деятельности) --- три критерия оценки долгосрочной экономической жизнеспособности (издержек жизненного цикла) и экономического развития;
                        \item Sociocultural and functional quality (Качество социокультурного и программного наполнения (среды)) --- восемь критериев социокультурного и функционального качества помогают оценивать здания с точки зрения здоровья, комфорта и удовлетворенности пользователей, а также основных аспектов функциональности;
                        \item Technical quality (Качество технических решений) --- семь критериев обеспечивают шкалу оценки технического качества с учетом соответствующих аспектов устойчивости;
                        \item Process quality (Качество процессов) --- девять критериев качества процесса направлены на повышение качества планирования и обеспечение качества строительства;
                        \item Site quality (Качество (строительной) площадки) --- четыре критерия качества площадки оценивают взаимовлияние окружающуей среды и проекта.
                    \end{enumerate} &
                    Оценка производится на основе показателей, описанных в межгосударственных стандартах Eurocode, действующих на территории Евросоюза.
                    Стандарты EN Eurocode делятся по аналогичным, принятым в системе DGNB, группам, и носят рекомендательный характер.
                    В некоторые стандарты закладываются математические модели, позволяющие провести физические расчёты и сформировать интегральные показатели строительно-инвестиционного проекта
                    в рамках Feasibility Studies \footnote{Технико-экономическое обоснование, проводимое на основе информационных моделей, полученных в ходе инженерных изысканий участка и градостроительного анализа территории на предмет ограничений и правового режима земельных участков.
                            В строительной практике США, стран Евросоюза и ряда других развитых стран является основным этапом принятия проектных решений и основой разработки строительной (<<рабочей>>) документации.
                            На этом этапе интегральные показатели информационной модели целостно охватывают проектируемый объект в контексте окружающей среды.}
                    с применением технологий информационного моделирования площадки строительства для принятия решений
                    Обязательные положения устанавливаются каждым из государств индивидуально, в соответствии с принципами гражданского общества и местного самоуправления \\ \hline

                % \hline
            \end{longtable}
        \end{center}
    
    % \KOMAoptions{paper=seascape}
    % \recalctypearea
\end{landscape}