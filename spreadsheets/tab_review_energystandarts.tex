\begin{landscape}
    
    % \KOMAoptions{paper=landscape}
    % \recalctypearea
        \begin{center}
            \begin{longtable}{|m{40mm}|p{40mm}|p{40mm}|p{55mm}|p{55mm}|}
               \caption{Общемировые стандарты и системы рейтинговой оценки энергетической эффективности зданий}
                \label{tab:review_energystandarts}
                \\ \hline
                % \multirow{2}{20mm}{№} &
                %     \multirow{2}{38mm}{Наименование улуса (района)} &
                %     \multirow{2}{38mm}{Улусный (районный) центр} &
                %     ГСОП, °С $\times$сут &
                %     \multicolumn{4}{c|}{$R_\text{огрконструкции}^\text{требуемое}\text{, м}{^2} \times \text{°С/Вт}$} \\
                % \cline{5-8}
                % &   &  & &  стен    &
                %     покрытий и перекрытий над проездами &
                %     Перекрытий чердачных над неотапливаемыми подпольями и подвалами &
                %     окон и балконных дверей, витрин и витражей  \\
                Наименование \mbox{стандарта} & Юрисдикция (страны действия) & Способ применения (добровольно/как часть НПА) &
                    Описание метрик (что оценивается) & Описание модели (как оценивается) \\

    
                \hline \endfirsthead
                % \cline{1-10}
                \subcaption{Продолжение таблицы~\ref{tab:review_energystandarts}}
                \\ \hline \endhead
                \hline \subcaption{Продолжение на след. стр.}
                \endfoot
                \hline \endlastfoot

                        % &       &   &    &
                BREEAM (\mbox{англ.}~Building  Research Establishment Environmental Assessment Method): рейтинговая  система  оценки «зеленых» зданий,разработанная в 1990~г. британской организацией BRE Global.&
                    &   &   & \\ \hline
                LEED (\mbox{англ.}~The Leadership in Energy \& Environmental Design): рейтинговая система оценки «зеленых» зданий, разработанная в 1998~г. Американским советом USGBC.&
                    &   &   & \\ \hline
                DGNB (\mbox{нем.}~Deutsche Gesellschaft fur Nachhaltiges Bauen): рейтинговая система оценки «зеленых» зданий, разработанная в 2007~г. Немецким советом по устойчивому строительству.&
                    &   &   & \\ \hline

                % \hline
            \end{longtable}
        \end{center}
    
    % \KOMAoptions{paper=seascape}
    % \recalctypearea
\end{landscape}