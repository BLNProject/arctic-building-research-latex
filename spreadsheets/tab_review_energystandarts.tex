\begin{landscape}
    
    % \KOMAoptions{paper=landscape}
    % \recalctypearea
        \begin{center}
            \begin{longtable}{|m{40mm}|p{40mm}|p{40mm}|p{55mm}|p{55mm}|}
               \caption{Общемировые стандарты и системы рейтинговой оценки энергетической эффективности зданий}
                \label{tab:review_energystandarts}
                \\ \hline
                % \multirow{2}{20mm}{№} &
                %     \multirow{2}{38mm}{Наименование улуса (района)} &
                %     \multirow{2}{38mm}{Улусный (районный) центр} &
                %     ГСОП, °С $\times$сут &
                %     \multicolumn{4}{c|}{$R_\text{огрконструкции}^\text{требуемое}\text{, м}{^2} \times \text{°С/Вт}$} \\
                % \cline{5-8}
                % &   &  & &  стен    &
                %     покрытий и перекрытий над проездами &
                %     Перекрытий чердачных над неотапливаемыми подпольями и подвалами &
                %     окон и балконных дверей, витрин и витражей  \\
                Наименование \mbox{стандарта} & Юрисдикция (страны действия) & Способ применения (добровольно/как часть НПА) &
                    Описание метрик (что оценивается) & Описание модели (как оценивается) \\

    
                \hline \endfirsthead
                % \cline{1-10}
                \subcaption{Продолжение таблицы~\ref{tab:review_energystandarts}}
                \\ \hline \endhead
                \hline \subcaption{Продолжение на след. стр.}
                \endfoot
                \hline \endlastfoot

                        % &       &   &    &
                BREEAM (\mbox{англ.}~Building  Research Establishment Environmental Assessment Method): рейтинговая  система  оценки «зеленых» зданий,разработанная в 1990~г. британской организацией BRE Global. &
                    &   &   & \\ \hline
                LEED (\mbox{англ.}~The Leadership in Energy \& Environmental Design): рейтинговая система оценки «зеленых» зданий, разработанная в 1998~г. Американским советом USGBC. &
                    &   &   & \\ \hline
                DGNB (\mbox{нем.}~Deutsche Gesellschaft fur Nachhaltiges Bauen): рейтинговая система оценки «зеленых» зданий, разработанная в 2007~г. Немецким советом по устойчивому строительству. &
                    Действует на территории стран Евросоюза &
                    Система DGNB оценивает не отдельные показатели, а общую производительность здания на основе критериев.
                    Если эти критерии выполняются превосходно, здание получает сертификат или предварительный сертификат (платина, золото, серебро или бронза) для существующей недвижимости.
                    DGNB продолжает развивать свою систему сертификации и адаптирует ее к национальным и международным стандартам и законодательству,
                    и, преемствуя парадигму системы Eurocode, носит рекомендательный характер &
                    Оценка производится по критериям, которые синхронизированны с 17-ю целями устойчивого развития ООН \cite{UN_17Goals} и техническими регламентами среды обитания человека,
                    закреплёнными в межгосударственных стандартах Eurocode. Критерии выделяются в следующие группы --- в соответствии с отраслями жизнедеятельности человека и предметными областями,
                    задействованными в развитии территорий:
                    \begin{enumerate}[1)]
                        \item Environmental quality (Качество среды) --- шесть критериев качества окружающей среды позволяют проводить оценку воздействия зданий на глобальную и местную окружающую среду, а также воздействия на ресурсы и образования отходов;
                        \item Economic quality (Качество хозяйственной деятельности) --- три критерия оценки долгосрочной экономической жизнеспособности (издержек жизненного цикла) и экономического развития;
                        \item Sociocultural and functional quality (Качество социокультурного и программного наполнения (среды)) --- восемь критериев социокультурного и функционального качества помогают оценивать здания с точки зрения здоровья, комфорта и удовлетворенности пользователей, а также основных аспектов функциональности;
                        \item Technical quality (Качество технических решений) --- семь критериев обеспечивают шкалу оценки технического качества с учетом соответствующих аспектов устойчивости;
                        \item Process quality (Качество процессов) --- девять критериев качества процесса направлены на повышение качества планирования и обеспечение качества строительства;
                        \item Site quality (Качество (строительной) площадки) --- четыре критерия качества площадки оценивают взаимовлияние окружающуей среды и проекта.
                    \end{enumerate} &
                    Оценка производится на основе показателей, описанных в межгосударственных стандартах Eurocode, действующих на территории Евросоюза.
                    Стандарты EN Eurocode делятся по аналогичным, принятым в системе DGNB, группам, и носят рекомендательный характер.
                    В некоторые стандарты закладываются математические модели, позволяющие провести физические расчёты и сформировать интегральные показатели строительно-инвестиционного проекта
                    в рамках Feasibility Studies \footnote{Технико-экономическое обоснование, проводимое на основе информационных моделей, полученных в ходе инженерных изысканий участка и градостроительного анализа территории на предмет ограничений и правового режима земельных участков.
                            В строительной практике США, стран Евросоюза и ряда других развитых стран является основным этапом принятия проектных решений и основой разработки строительной (<<рабочей>>) документации.
                            На этом этапе интегральные показатели информационной модели целостно охватывают проектируемый объект в контексте окружающей среды.}
                    с применением технологий информационного моделирования площадки строительства для принятия решений
                    Обязательные положения устанавливаются каждым из государств индивидуально, в соответствии с принципами гражданского общества и местного самоуправления \\ \hline

                % \hline
            \end{longtable}
        \end{center}
    
    % \KOMAoptions{paper=seascape}
    % \recalctypearea
\end{landscape}